% !TeX root = ../notes.tex

\section{Data Spaces und Data Ecosystems}

\subsection{Datenaustausch für Lieferketten}

\textbf{Data Sharing}
\begin{itemize}
    \item Data Sharing seit über 40 Jahren Bestandteil der Forschung
    \begin{itemize}
        \item Data Sharing: Prozess bei dem anderen Zugriff auf Daten gegeben wird, auf welche sie sonst nicht selbst Zugriff hätten
    \end{itemize}
    \item Daten für nahezu alle Geschäftsprozesse notwendig
    \item früh: versch. Vorteile von \emph{Information Partnerships} erkannt, bspw. geteilte Kosten, Verteilung von Überkapazitäten durch gemeinsame Nutzung von Kundendaten
    % \item hohe Integrationskosten, nur für reiche Unternehmen zugänglich
    \item v.a. für Koordination von Lieferketten essentiell
    \item Entgegenwirken von Fluktuationen in der Lieferkette nur mit Datenbasis möglich~\cite{mollerIndustrialDataEcosystems2024}
\end{itemize}

\vspace{1cm}

\textbf{Lieferketten}
\begin{itemize}
    \item starke Einschränkung der geteilten Daten und Informationen
    \item sehr \emph{spezifische} Daten in \emph{bilateralen} Beziehungen
    \item Data Sharing nur für bestimmten Zweck: Eindämmung negativer Effekte in Lieferketten
    \item Angst vor Missbrauch oder Verletzung der Vertraulichkeit
    \item \emph{Information Asymmetry}
    \item keine Daten teilen $\to$ Verluste, ineffiziente Lieferketten $\to$ keine Option
    \item Data Sharing benötigt Vertrauen in Wahrheitsgehalt geteilter Daten~\cite{mollerIndustrialDataEcosystems2024}
\end{itemize}


\subsection{Industrielle Data Ecosystems}

\begin{itemize}
    \item Erscheinung digital transformierter Systeme an Unternehmen
    \item alternative Sicht zu zwischenbetrieblichem Data Sharing
    \item dynamisches Data Sharing basierend auf gemeinsamen Werten (statt nur Geschäftsprozessen)
    \item Entwicklung um einen geteilten Zweck, dynamisch
    \item Gleichgewicht zwischen erhaltenem und gegebenem Aufwand (vgl. Information Asymmetry)
    \item autonome multilaterale Akteure
    \begin{itemize}
        \item ergänzen sich gegenseitig
        \item schaffen Netzwerkeffekte
    \end{itemize}
    \item zwischenbetriebliches Data Sharing + Anlehnung an biologische Ökosysteme $\to$ \emph{Data Ecosystems}
    \begin{itemize}
        \item Akteure interagieren und kooperieren
        \item \enquote{Finden, Archivieren, Veröffentlichen, Konsumieren oder Wiederverwenden von Daten}
        \item Förderung von Innovation, Wertschöpfung und Schaffung neuer Geschäfts
        \item[$\Rightarrow$] Erstellen, Verwalten und Aufrechterhalten von Anreizen für Data Sharing
    \end{itemize}
    \item Kern: Beitrag jedes Akteurs
    \begin{itemize}
        \item Data User: Verwendung von Daten
        \item Data Provider: Bereitstellung / Senden von Daten
        \item Data Intermediary: Schnittstellen~\cite{mollerIndustrialDataEcosystems2024}
    \end{itemize}
\end{itemize}


\subsection{Data Sharing Infrastructure}

\begin{itemize}
    \item Data Sharing benötigt technische Infrastruktur
    \item Kategorien: basierend auf Data Intermediaries oder \emph{Inter-Organisational Information Systems} (IOIS)
\end{itemize}

\vspace{1cm}

\textbf{IOIS}
\begin{itemize}
    \item v.a. in Lieferketten
    \item Verbindung individueller Systeme von Betrieben (bspw. ERP) für bessere Integration und automatisiertem Data Sharing
    \item meist bilateral oder multilaterale Netzwerke basierend auf einem zentralen Akteur
    \item Skalierung sehr schwierig, bspw. bzgl. Datenschutz und Kontrolle (Datensouveränität)
\end{itemize}

\vspace{1cm}

\textbf{Data Spaces}
\begin{itemize}
    \item Vereinen von DI und IOIS (vgl. \autoref{fig:dual-nature-ds})
    \item geteilter Raum für Unternehmen zur Suche nach vertrauenswürdigen Quellen
    \item Antrieb für organisationsübergreifende Optimierung und Innovation
    \item ähnlich zu DI als Vermittler zwischen Data User und Data Provider
    \item Verwendung von \emph{Data Connectors}, ermöglichen bilaterales Data Sharing (vgl. IOIS)
    \item Connectors: technisch sichergestellte Datensouveränität
    \item dezentralisiert, keine zentrale Datenspeicherung, Data Provider
    \item Data Sharing erst nach erfolgreicher Verhandlung
    \item Generierung eines \emph{Trusted Pool}s an DE-Akteuren~\cite{mollerIndustrialDataEcosystems2024}
\end{itemize}

\begin{figure}
    \includegraphics[width=\textwidth]{../shared/assets/möller_dual_nature_ds.png}
    \caption{Duale Natur von Data Spaces als Data Intermediaries und IOIs~\cite{mollerIndustrialDataEcosystems2024}}
    \label{fig:dual-nature-ds}
\end{figure}

\begin{figure}
    \includegraphics[width=\textwidth]{../shared/assets/möller_iois_di_ds.png}
    \caption{Data Sharing mit IOIS, Data Intermediaries und Data Spaces~\cite{mollerIndustrialDataEcosystems2024}}
\end{figure}


\subsection{Data Spaces in Data Ecosystems}

\begin{itemize}
    \item Data Spaces als Motor für Data Ecosystems (s.o.)
    \item dezentralisierte Daten"=Infrastruktur
    \item entworfen für Datenaustausch über mehre Unternehmen hinweg
    \item möglich durch Mechanismen für sicheres und vertrauenswürdiges Data Sharing
    \item Garantie von Datensouveränität
    \begin{itemize}
        \item Data Provider entscheidet über Zugriffskontrolle und Verwendung von geteilten Daten
    \end{itemize}
    \item flexible Formen, erlaubt bestimmter Menge an Mitgliedern Zugriff auf sicheren, vertrauenswürdigen Raum für Data Sharing
    \item Einbettung in Data Ecosystems möglich~\cite{mollerIndustrialDataEcosystems2024}
\end{itemize}

\vspace{1cm}

\begin{itemize}
    \item \emph{Data Space Member}: Akteur, welcher direkt (technisch) am Datenaustausch über DS beteiligt ist
    \item \emph{Data Ecosystem Member}: Teil des Ökosystems, aber nicht direkt am Datenaustausch durch DS beteiligt
    \begin{itemize}
        \item Beitrag zu Daten
        \item Zugriff durch \emph{Data Space Connectors}
        \begin{itemize}
            \item Schnittstelle zwischen internen Systemen der DSM und des DS an sich
            \item Erweiterungen möglich, bspw. zur Interpretation und technischer Durchsetzung von \emph{Data Usage Policies}~\cite{mollerIndustrialDataEcosystems2024}
        \end{itemize}
    \end{itemize}
\end{itemize}

\begin{figure}
    \includegraphics[width=\textwidth]{../shared/assets/möller_data_ecosystem.png}
    \caption{Data Ecosystem am Beispiel von Catena-X und Mobility Data Space~\cite{mollerIndustrialDataEcosystems2024}}
\end{figure}

\textbf{Eigenschaften von Data Spaces}
\begin{itemize}
    \item verteilt
    \begin{itemize}
        \item DS sind verteilt by Design
        \item benötigen keine physikalische Datenintegration
        \item Daten bleiben bei Quelle, Zugang nur wenn notwendig
    \end{itemize}

    \item kein einheitliches Daten"=Schema notwendig
    \begin{itemize}
        \item Datenintegration auf semantischer Ebene
        \item bspw. durch einheitliche \emph{Vocabularies}
    \end{itemize}
    
    \item Datenredundanz
    \begin{itemize}
        \item durch verteilte Architektur
        \item Daten können an versch. Orten ko-existieren
    \end{itemize}
    
    \item verschachtelt und überlappend
    \begin{itemize}
        \item DS können verschachtelt oder überlappend sein
        \item Data Provider und Data User können Mitglieder versch. DS sein
        \item Data Sharing zwischen DS~\cite{mollerIndustrialDataEcosystems2024}
    \end{itemize}
\end{itemize}

\vspace{1cm}

\textbf{Data Ecosystems}
\begin{itemize}
    \item DE entstehen um einen oder mehrere föderierte DS
    \item repräsentieren gesamte kollaborative DS-Aktivitäten
    \item Realisierung gemeinsamer Ziele der Akteure
    \item DE können mehrere DS umfassen (techn. Integration mittels Schnittstellen)
    \item Ziel: überlappende DE mit verbundenen DS $\to$ verhindern Bildung von großen Daten"=Silos
    \item Governance"=Maßnahmen über alle Abstraktionsschichten möglich (DE, DS, Use Case)
    \begin{itemize}
        \item bspw. rechtliche Anforderungen, Regeln von DS, bilaterale Regeln
    \end{itemize}
\end{itemize}

\begin{figure}
    \includegraphics[width=\textwidth]{../shared/assets/möller_data_ecosystems.png}
    \caption{Verschiedene Szenarien für Data Ecosystems~\cite{mollerIndustrialDataEcosystems2024}}
\end{figure}
