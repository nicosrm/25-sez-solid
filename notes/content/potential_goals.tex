% !TeX root = ../notes.tex

\section{Potenziale und Ziele}

% \item[$\Rightarrow$] V: Aufwandsminimierung $\to$ Kostenminimierung $\to$ Steigerung der Wirtschaftlichkeit
% $\to$ Steigerung des Wettbewerbs $\to$ Förderung von Innovation und Zusammenarbeit

\textbf{Vertrauen?}
\begin{itemize}
    \item Warum kein Vertrauen?
    \begin{itemize}
        \item bspw. Angst vor Datenmissbrauch oder unberechtigter Weitergabe
        \item Was würde helfen?
        \begin{itemize}
            \item Nutzende verwalten Daten unabhängig von Anwendungen
            \item Daten gehören den Nutzenden, Kontrolle über Daten(zugang)
        \end{itemize}
        \item[$\Rightarrow$] Schaffen durch Datensouveränität in Kombination mit Datenschutz
    \end{itemize}

    \item Wem kann ich vertrauen?
    \begin{itemize}
        \item ad-hoc: Wirklich die Person? Wem kann ich vertrauen?
        \item Authentifizierungsverfahrung und Vertrauen in Wahrheitsgehalt der geteilten Daten notwendig
        \item Wie erreichen?
        \begin{itemize}
            \item Authentifizierungsentscheidungen basierend auf Netzwerk
            \item[$\Rightarrow$] \emph{Web of Trust}
        \end{itemize}
        \item[$\Rightarrow$] ad-hoc Teilen von Daten auf dieser Basis möglich
    \end{itemize}
    
    \item technische Mechanismen zur Einhaltung gesetzlicher Vorgaben
    \begin{itemize}
        \item Möglichkeit: unterschiedliche Datenspeicher je nach Kontext, Grad der Kontrollmöglichkeiten, gesetzlichen Gegebenheiten etc.
        \item bspw. Medizin- getrennt von Bewerbungsdaten
        \item[$\to$] physische / logische Trennung je nach Anwendungsfall
    \end{itemize}
    \item Ziel: Skalierbarkeit und Effizienz
    \item Ziel: sicherer, effizienter Austausch von Daten
    \item[$\Rightarrow$] Dezentralisierung
\end{itemize}

\vspace{1em}

\textbf{Aktualität und Verfügbarkeit von Daten?}
\begin{itemize}
    \item Warum veraltete Daten und schlechte Verfügbarkeit?
    \item Angst vor Datenmissbrauch oder unberechtigter Weitergabe
    \item[$\Rightarrow$] Datensouveränität und Datenschutz
    \item[$\to$] höhere Wahrscheinlichkeit zur Speicherung von mehr und diverseren Daten
    \item[$\Rightarrow$] Aktualität und Verfügbarkeit Daten
\end{itemize}

\vspace{1em}

\textbf{Effizienz und Geschwindigkeit? Kostenreduktion?}
\begin{itemize}
    \item Kostenfaktor: wdh. Schritte bei Integration, hohe Entwicklungskosten durch Re-Implementierung von Standardfunktionalitäten
    \item erreichbar durch Auslagerung von Standardfunktionalitäten, Wiederverwendbarkeit, Interoperabilität
    \begin{itemize}
        \item Entkopplung von Anwendungen und Daten
        \item einfacher Wechsel zwischen Anwendungen und Datenspeichern
        \item einfache Anwendungsentwicklung
    \end{itemize}
    \item[$\Rightarrow$] Automatisierung, Wiederverwendbarkeit $\to$ Senkung von Kosten
    \item mehrere Akteure können durch Interoperabilität zusammenarbeiten
    \item[$\Rightarrow$] mehr Möglichkeiten zur Zusammenarbeit
    \item[$\Rightarrow$] Verschnellerung des Prozesses
    \item[$\Rightarrow$] \emph{ad-hoc} Zusammenschaltung von Geschäftsprozessen / Daten- / Anwendungsintegration
\end{itemize}

\vspace{1em}

\textbf{Zugänglichkeit?}
\begin{itemize}
    \item Aufwandsminimierung $\to$ Kostenminimierung $\to$ Steigerung der Wirtschaftlichkeit
    \item einfacher Wechsel zwischen Anwendungen und Datenspeichern
    \item[$\to$] niedrigere Einstieghürden für Entwicklung neuer Anwendungen
    \item[$\Rightarrow$] Zugänglichkeit
    \item[$\Rightarrow$] Wettbewerb, Innovation
\end{itemize}

\vspace{1em}

\textbf{Wie erreichen wir das?}
\begin{itemize}
    \item V: Zusammenarbeit verschiedener Akteure in einem vertrauenswürdigen, offenen, interoperablen System
    \item bestimmte \emph{Enabler} notwendig, um in richtige Richtung zu kommen
\end{itemize}
