% !TeX root = ../notes.tex

\section{Vision}

\begin{itemize}
    \item Daten als wertvolle, strategische, teilbare und vervielfältigbare Ressource statt als \enquote{Nebenprodukt}~\cite{mollerIndustrialDataEcosystems2024}
    \begin{itemize}
        \item mehr Möglichkeiten für Wettbewerbsfähigkeit und Diversifizierung~\cite{mollerIndustrialDataEcosystems2024}
        \item \enquote{Leading management consultancy Gartner predicts that \enquote{organizations that promote data sharing will outperform their peers on most business value metrics}}~\cite{mollerIndustrialDataEcosystems2024}
    \end{itemize}
    \item Unternehmen liefern Informationen an Lieferanten / Kunden, notwendig für Koordination und Erfüllung rechtlicher Rahmenbedingungen~\cite{mollerIndustrialDataEcosystems2024}
    \item zwischenbetrieblicher Informationsaustausch ~\cite{mollerIndustrialDataEcosystems2024}
    \item viele Bedenken beim Teilen, bspw. Angst vor unberechtigter Weitergabe oder Missbrauch von Daten (bspw. \emph{Business Secrets}), Kontrollverlust~\cite{mollerIndustrialDataEcosystems2024}
    \item mangelndes Vertrauen beim \emph{Data Sharing}
    \item meist nur bilateral, für bestimmten Zweck~\cite{mollerIndustrialDataEcosystems2024}
    \item[$\Rightarrow$] V: Vertrauen, multi"=laterales Data Sharing
\end{itemize}

\vspace{1cm}

\begin{itemize}
    \item Datenintegration essentiell für Geschäftsprozesse
    \item ständig neue Geschäftsprozessen, Integration mehrerer Projekte
    \item aktuell: \emph{Extract, Transform, Load} (ETL) Prozesse
    \item ähnliche Strukturen müssen oft neu implementiert werden, wiederholte Schritte bei Integration
    \item zeitintensiv, schnelle Weiterentwicklung $\to$ nach Abschluss haben sich Modelle / Daten ggf. schon wieder verändert
    \item kostenintensiv, inkonsistent
    \item hohe Einstiegsbarriere für neue Akteure
    \item[$\Rightarrow$] V: Zusammenschaltung von Geschäftsprozessen / Daten- / Anwendungsintegration soll ad-hoc möglich sein
    \item[$\Rightarrow$] V: stets verfügbare, aktuelle Daten
    \item[$\Rightarrow$] V: Aufwandsminimierung $\to$ Kostenminimierung $\to$ Steigerung der Wirtschaftlichkeit
    \item[$\Rightarrow$] V: niedrige Einstiegshürden für neue Akteure $\to$ Steigerung des Wettbewerbs $\to$ Förderung von Innovation und Zusammenarbeit
\end{itemize}

\vspace{1cm}

\begin{itemize}
    \item Daten als wertvolles Wirtschaftsgut
    \item Speicherung von Daten \emph{en masse} $\to$ große Datensilos
    \item Abgabe von Kontrolle \emph{oder} Privatsphäre $\to$ Entscheidung zwischen beiden notwendig $\to$ Bedenken beim Data Sharing, ggf. zurückhaltend
    \item mangelnde Kooperation / Vertrauen $\to$ mehrfache Speicherung derselben Daten bei verschiedenen Unternehmen $\to$ inkonsistente, veraltete Daten
    \item[$\Rightarrow$] V: Verfügbarkeit von aktuellen, konsistenten Daten
    \item[$\Rightarrow$] V: Datenschutz und Datensouveränität zusammen
\end{itemize}
