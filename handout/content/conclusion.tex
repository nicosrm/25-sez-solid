% !TeX root = ../handout.tex

\section{Fazit}

In diesem Handout wurden Data Spaces vorgestellt, welche dezentrale, multilaterale Informationssysteme anstreben, um ein vertrauenswürdiges Data Sharing unter Garantie von Datensouveränität ermöglichen.
Eine mögliche Umsetzung basiert auf dem offenen Standard Solid, welcher die Verwaltung und den Austausch von Daten und Identitäten definiert.
Daten werden dezentral in Pods gespeichert, wobei die volle Kontrolle über den Zugang bei Nutzenden liegt.
Durch Standardisierung und Interoperabilität können Datenspeicher unabhängig von Anwendungen gewechselt werden.
Authentifizierungsentscheidungen können ad-hoc über ein Web of Trust getroffen werden.
Somit soll ein vertrauenswürdiges Teilen von aktuellen und konsistenten Daten als Basis für effiziente Integrationen ermöglicht werden.
Zukünftig könnte dies in Richtung vollständige Digitalisierung, Metaprogrammierung mit agnostischen Daten sowie eine Nutzerzentrierung von Anwendungen führen.


% \begin{itemize}
%     \item Vision: vertrauenswürdiges Data Sharing, aktuelle und konsistente Daten, Zugänglichkeit, effiziente Integrationen, Förderung von Innovation und Kooperation

%     \item Data Spaces: dezentrale, multilaterale Informationssysteme; ermöglichen vertrauenswürdiges Data Sharing unter Datensouveränität

%     \item Social Linked Data: offener Standard zum Verwalten und Teilen von Daten im Web; Verwendung zum Erschaffen von Data Spaces
    
%     \item dezentrale Datenspeicherung in Pods; volle Kontrolle durch Nutzende; Wechsel von Datenspeicher unabhängig von Anwendung
    
%     Daten werden dezentral in Pods gespeichert, wobei Nutzende die volle Kontrolle über den Zugang haben.
%           Durch Standardisierung und Interoperabilität können Datenspeicher unabhängig von Anwendungen gewechselt werden.
    
%     \item Daten sind in RDF strukturiert, sodass diese automatisierbare Semantik enthalten, und verwandten Ressourcen verlinkt sind (Linked Data).
    
%     \item Zukunft: Digitalisierung, datensouveränes Data Sharing, Mapping statt ETL, Data  Mesh statt Data Lake, Metaprogrammierung mit agnostischen Daten, Nutzerzentrierung von Anwendungen
% \end{itemize}
