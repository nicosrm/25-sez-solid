% !TeX root = ../handout.tex

\section{Einschätzung und Zukunft}

Die Verwendung von Solid vereint viele Ziele, um die einleitend genannte Vision näher an die Realität zu bringen.
So existiert ein Ansatz zur Gewährleistung von Datensouveränität und Datenschutz, auch wenn noch Konzepte notwendig sind, welche die Zwischenspeicherung durch Data User und somit die Bildung großer Datensilos verhindert.
Mittels Web of Trust können Authentifizierungsentscheidungen schnell und effizient getroffen werden, wodurch vertrauenswürdig Daten geteilt werden können.
Jedoch sind Ansätze gegen Kompromittierung notwendig.

Durch Standardisierung, Interoperabilität und Wiederverwendbarkeit werden Anwendungen und Daten entkoppelt, wodurch ein Wechsel ohne aufwendige Migrationen ermöglicht wird.
Gekoppelt mit der Auslagerung von Standardfunktionalitäten können Entwicklungskosten und somit Einstiegsbarrieren für neue Unternehmen gesenkt werden.
Dadurch ist mehr Wettbewerb und Innovation möglich.

Insgesamt ist das Ziel, ad-hoc Daten"= und Anwendungsintegrationen zu ermöglichen, wofür statt aufwendigen ETL"=Prozessen nur ein Mapping der Datenstrukturen vorgenommen werden muss.
Daten können von Domänen abstrahiert werden, da eine dynamische Interpretation durch die Semantik aus den RDF"=Daten möglich ist, was in Richtung Metaprogrammierung führen könnte.
Allerdings müssen die Daten vorher in entsprechenden, digitalen Formaten vorliegen.

Da die Datenstrukturen nicht mehr von den Anwendungen abhängen, wird eine andere Art und Weise der Unterscheidung auf dem Markt notwendig sein, da Unternehmen ihre Kund:innen nicht mehr durch proprietäre Datenformate halten können.
Eine Nutzerzentrierung sowie der Qualitätsanspruch könnten zunehmen, da Nutzende sonst problemlos zu anderen Alternativen wechseln könnten.
Durch eine niedrigere Einstiegshürde in den Markt würde dies begünstigt werden.
