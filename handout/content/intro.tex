% !TeX root = ../handout.tex

\section{Einleitung und Motivation}
% (Beschreibung von Kontext, Problemen, Anforderungen und Zielen)

\begin{itemize}
    \item aktueller Stand
    \begin{itemize}
        \item zentralisiert
        \item Datensilos, Datenmissbrauch, veraltete Daten
    \end{itemize}
    
    \item industrieller Kontext
    \begin{itemize}
        \item Daten als strategische Ressource
        \item Supply Chain (Act) Bedenken beim Teilen von Daten
    \end{itemize}

    \item Vision
    \begin{itemize}
        \item Datensouveränität, Datenschutz (privat, betrieblich)
        \item dezentralisierte Plattform, Interoperabilität
        \item keine Datensilos, weniger Entwicklungsaufwand, einfacher Wechsel
        \item \emph{Web of Trust}, Zero"=Trust"=Architektur
        \item einfache, sichere Authentifizierung
        \item universelle, integrierte, automatisierte, konforme Lösung für E2E"=Data"=Value"=Chains
              $\Rightarrow$ Steigerung der Wirtschaftlichkeit
    \end{itemize}

    \item \hl{TODO: hier auch Potenziale / Ziele?}

    \item \emph{Data Spaces} als Lösung für Probleme durch technische Implementierung von Datensouveränität
    \item Struktur des Handouts
\end{itemize}

\cite{sambraSolidPlatformDecentralized2016,bothSolidBasedB2BData2025,mecklerWebLinkedData2023,mollerIndustrialDataEcosystems2024}

% (kurze Zusammenfassung der Struktur der Belegarbeit)
% Diese Arbeit ist folgendermaßen strukturiert. 
% In Kapitel ... 
% ...
% ...
% Abschließend ...
