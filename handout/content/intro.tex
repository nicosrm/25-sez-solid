% !TeX root = ../handout.tex

\section{Einleitung und Motivation}

Seit über 40 Jahren ist das Konzept des \emph{Data Sharing} Bestandteil der Forschung.
Data Sharing beschreibt dabei den Prozess, bei dem Dritten Zugriff auf Daten gewährt wird, auf welche diese sonst keinen Zugriff hätten.
Das Internet und die Einführung von Smartphones ermöglicht es, nahezu sofort Daten zu erhalten und weiter zu verteilen.
Auch im betrieblichen Kontext sind Daten essenziell.
Sie sind für nahezu alle Geschäftsprozesse notwendig, sodass sie vom \enquote{Nebenprodukt} zur \emph{strategischen Ressource} geworden sind~\cite{mollerIndustrialDataEcosystems2024}.
Es gibt viele Bedenken beim Teilen von Daten, wie bspw. der Angst vor unberechtigter Weitergabe (Geschäftsgeheimnisse), dem Missbrauch von Daten oder vor Kontrollverlust~\cite{mollerIndustrialDataEcosystems2024}.
Eine Vertrauensbasis beim Data Sharing fehlt, um ein multi"=laterales Datennetz zu spannen.

Um Daten zwischen verschiedenen Akteuren zu teilen, ist eine Datenintegration notwendig, welche essenziell für den Erfolg von Geschäftsprozessen ist.
Solche Integrationen sind oftmals aufwendig und zeitintensiv.
Nach Abschluss haben sich die Daten, Modell und Projekte teilweise schon wieder verändert, sodass Inkonsistenzen und hohe Kosten durch die wiederholte Ausführung von Schritten entstehen.
Durch hohe Kosten entsteht eine hohe Einstiegsbarriere für neue Akteure, wodurch die Zugänglichkeit und Innovationsfähigkeit eingeschränkt wird.
Ein effizientes, schnelles und günstiges Verfahren, welches für alle zugänglich, sowie stets verfügbare und konsistente Daten schafft, ist notwendig.

Da Daten als wertvolles Wirtschaftsgut zu betrachten ist, werden diese \emph{en masse} gespeichert.
Aufgrund von mangelndem Vertrauen und dadurch mangelnder Kooperation, werden dieselben Daten mehrfach an verschiedenen Orten gespeichert.
Da Nutzende sich oft zwischen Kontrolle über ihre Daten und Privatsphäre entscheiden müssen, werden diese oft nur zurückhaltend geteilt.
Somit entstehen mehrere große Datensilos, welche inkonsistent und teils veraltete Daten enthalten.
Wünschenswert wäre die Verfügbarkeit von aktuellen, konsistenten Daten sowie die Kombination Datenschutz und Zugriffskontrolle (vgl. Datensouveränität).

An dieser Stelle setzt das Konzept der \emph{Data Spaces} an. Das Konzept spricht diese Probleme an, in dem es einen multi"=lateralen, sicheren und vertrauenswürdigen Datenaustausch ermöglicht, welches Datensouveränität garantiert~\cite{mollerIndustrialDataEcosystems2024}.
Der Begründer des \emph{World Wide Web}s, Tim Berners"=Lee, stellte dazu 2016 den \emph{Social Linked Platform} (Solid) Standard vor, welches ein Fundament für offene, dezentralisierte Netzwerke für einen souveränen Datenaustausch ermöglichen möchte~\cite{mecklerWebLinkedData2023}.

% Auf den folgenden Seiten wird zunächst das Konzept der Data Spaces hergeleitet und erklärt. Anschließend wird Solid als mögliche Umsetzung dargelegt und im Vergleich zum aktuellen, zentralisierten Ansatz eingeschätzt. Schlussendlich wird eine potenzielle Zukunft des Software Engineerings skizziert und ein Fazit gezogen.
