% !TeX root = presentation.tex

% !TeX root = ../notes.tex

\documentclass[
    a4paper,
    12pt,
]{article}

% !TeX root = ../handout.tex

\usepackage[
    top=3cm,
    bottom=3cm,
    left=2.5cm,
    right=2.5cm
]{geometry}

\usepackage[T1]{fontenc}
\usepackage[utf8]{inputenc}

\usepackage[english,ngerman]{babel}

\usepackage[
  final,
  colorlinks=true,
  linkcolor=blue
]{hyperref}
\usepackage[htt]{hyphenat}  % htt -> allow to hyphenate TT text

\usepackage[backend=biber, style=alphabetic]{biblatex}
\addbibresource{../shared/references.bib}

\usepackage{parskip}
\usepackage{blindtext}

\usepackage{subcaption}
\usepackage{graphicx}
% \usepackage{subfigure}
\usepackage{caption}
\usepackage{float}

\usepackage{csquotes}

% vertical padding in tables
\renewcommand{\arraystretch}{1.2}

\usepackage{soul}


\usepackage[figurename=Fig.]{caption}
\makeatletter
\renewcommand{\fnum@figure}{Abb. \thefigure}
\makeatother
\addto\captionsngerman{\renewcommand{\figurename}{Abb.}}

% !TeX root = ../presentation.tex

\title[Die Zukunft des Datenmanagements? Data Spaces am Beispiel von Solid]{Die Zukunft des Datenmanagements?\\Data Spaces am Beispiel von Solid}

% \subtitle[SUBTITLE]
%     {FORMATTED\\SUBTITLE}

\date{08. Juli 2025}
\author[N. Schramm]{Nico Schramm}

\titlegraphic{\flushright\includegraphics[width=2cm]{./assets/htwk_logo.png}}

\newcommand{\module}{Oberseminar: \enquote{Innovative Methoden für Software Engineering der Zukunft}}
\newcommand{\prof}{Prof. Dr. Andreas Both}

\newcommand{\faculty}{Fakultät Informatik und Medien}
\newcommand{\university}{Hochschule für Technik, Wirtschaft und Kultur Leipzig}

\institute{\module\\
  \prof\\~\\
  \faculty\\
  \university}

\newcommand{\repourl}{https://github.com/nicosrm/25-sez-solid}

\hypersetup{
  % PDF metadata
  pdfauthor={Nico Schramm},
  pdftitle={Data Spaces am Beispiel von Solid},
  pdfsubject={Präsentation},
  % hyperref colouring
  colorlinks,
  allcolors=.,
  urlcolor=blue,
  citecolor=cyan
}


\makeatletter
\hypersetup{
    pdfauthor={\@author},
    pdftitle={\@title},
    pdfsubject={\subject},
    citecolor=blue
}
\makeatother


\begin{document}

\maketitle

% !TeX root = ../handout.tex

\section{Einleitung und Motivation}

Seit über 40 Jahren ist das Konzept des \emph{Data Sharing} Bestandteil der Forschung.
Data Sharing beschreibt dabei den Prozess, bei dem Dritten Zugriff auf Daten gewährt wird, auf welche diese sonst keinen Zugriff hätten (vgl. Beispiel aus \autoref{fig:example-data-donation}).
Das Internet und die Einführung von Smartphones ermöglicht es, nahezu sofort Daten zu erhalten und weiter zu verteilen.
Auch im betrieblichen Kontext sind Daten essenziell.
Sie sind für nahezu alle Geschäftsprozesse notwendig, sodass sie sich vom \enquote{Nebenprodukt} zur \emph{strategischen Ressource} entwickelten~\cite{mollerIndustrialDataEcosystems2024}.
Es gibt viele Bedenken beim Teilen von Daten, wie bspw. der Angst vor unberechtigter Weitergabe (Geschäftsgeheimnisse), Missbrauch von Daten oder Kontrollverlust~\cite{mollerIndustrialDataEcosystems2024}.
Eine Vertrauensbasis fehlt, um ein multi"=laterales Datennetz zu spannen.

\begin{figure}
    \includegraphics[width=\textwidth]{./assets/example_horizontal.drawio.pdf}
    \caption{Beispiel für Data Sharing: Datenspende zur Weiterverwendung für Forschung}
    \label{fig:example-data-donation}
\end{figure}

Um Daten zwischen verschiedenen Akteuren zu teilen, ist oft eine Datenintegration, bspw. via \emph{Extract, Transform, Load} (ETL) notwendig, welche essenziell für den Erfolg von Geschäftsprozessen ist.
Solche Integrationen sind oftmals aufwendig und zeitintensiv.
Nach Abschluss haben sich die Modelle und Projekte teilweise schon wieder verändert, sodass Inkonsistenzen und hohe Kosten durch die wiederholte Ausführung von Schritten entstehen.
Durch hohe Kosten liegt die Einstiegsbarriere für neue Akteure hoch, wodurch die Zugänglichkeit und Innovationsfähigkeit eingeschränkt wird.
Ein effizientes, schnelles und günstiges Verfahren, welches für alle zugänglich, sowie stets verfügbare und konsistente Daten schafft, ist notwendig.

Da Daten als wertvolles Wirtschaftsgut zu betrachten ist, werden diese \emph{en masse} gespeichert.
Aufgrund von mangelndem Vertrauen und dadurch mangelnder Kooperation, werden dieselben Daten mehrfach an verschiedenen Orten gespeichert.
Da Nutzende sich oft ihre Daten und Privatsphäre nicht kontrollieren können, werden diese oft nur zurückhaltend geteilt.
Somit entstehen mehrere große Datensilos, welche inkonsistent und teils veraltete Daten enthalten.
Wünschenswert wäre die Verfügbarkeit von aktuellen, konsistenten Daten sowie die Kombination Datenschutz und Zugriffskontrolle (vgl. Datensouveränität).

An dieser Stelle setzt das Konzept der \emph{Data Spaces} an. Dieses adressiert diese Probleme, in dem es einen multi"=lateralen, sicheren und vertrauenswürdigen Datenaustausch ermöglicht, welches Datensouveränität garantiert~\cite{mollerIndustrialDataEcosystems2024}.
Der Begründer des \emph{World Wide Web}s, Tim Berners"=Lee, stellte dazu 2016 den Solid"=Standard (ehem. \emph{Social Linked Data}) vor, welches ein Fundament für offene, dezentralisierte Netzwerke für einen souveränen Datenaustausch ermöglichen möchte~\cite{mecklerWebLinkedData2023}.


\begin{frame}{Gliederung}
    \tableofcontents
\end{frame}

% !TeX root = ../notes.tex

\section{Vision}

\begin{itemize}
    \item Daten als wertvolle, strategische, teilbare und vervielfältigbare Ressource statt als \enquote{Nebenprodukt}~\cite{mollerIndustrialDataEcosystems2024}
    \begin{itemize}
        \item mehr Möglichkeiten für Wettbewerbsfähigkeit und Diversifizierung~\cite{mollerIndustrialDataEcosystems2024}
        \item \enquote{Leading management consultancy Gartner predicts that \enquote{organizations that promote data sharing will outperform their peers on most business value metrics}}~\cite{mollerIndustrialDataEcosystems2024}
    \end{itemize}
    \item Unternehmen liefern Informationen an Lieferanten / Kunden, notwendig für Koordination und Erfüllung rechtlicher Rahmenbedingungen~\cite{mollerIndustrialDataEcosystems2024}
    \item zwischenbetrieblicher Informationsaustausch ~\cite{mollerIndustrialDataEcosystems2024}
    \item viele Bedenken beim Teilen, bspw. Angst vor unberechtigter Weitergabe oder Missbrauch von Daten (bspw. \emph{Business Secrets}), Kontrollverlust~\cite{mollerIndustrialDataEcosystems2024}
    \item mangelndes Vertrauen beim \emph{Data Sharing}
    \item meist nur bilateral, für bestimmten Zweck~\cite{mollerIndustrialDataEcosystems2024}
    \item[$\Rightarrow$] V: Vertrauen, multi"=laterales Data Sharing
\end{itemize}

\vspace{1cm}

\begin{itemize}
    \item Datenintegration essentiell für Geschäftsprozesse
    \item ständig neue Geschäftsprozessen, Integration mehrerer Projekte
    \item aktuell: \emph{Extract, Transform, Load} (ETL) Prozesse
    \item ähnliche Strukturen müssen oft neu implementiert werden, wiederholte Schritte bei Integration
    \item zeitintensiv, schnelle Weiterentwicklung $\to$ nach Abschluss haben sich Modelle / Daten ggf. schon wieder verändert
    \item kostenintensiv, inkonsistent
    \item hohe Einstiegsbarriere für neue Akteure
    \item[$\Rightarrow$] benötigen effizientes, schnelles und günstiges Verfahren, was für alle zugänglich ist
    \item[$\Rightarrow$] V: effiziente und schnelle Prozesse
    \item[$\Rightarrow$] V: stets verfügbare, aktuelle, konsistente Daten
    \item[$\Rightarrow$] V: niedrige Einstiegshürden für neue Akteure, zugänglicher Markt unabhängig von Größe, Branche, Kapital etc.
    \item[$\Rightarrow$] Umfeld etablieren, in dem innovative Lösungen möglich sind / begünstigt werden
\end{itemize}

\vspace{1cm}

\begin{itemize}
    \item Daten als wertvolles Wirtschaftsgut
    \item Speicherung von Daten \emph{en masse} $\to$ große Datensilos
    \item Abgabe von Kontrolle \emph{oder} Privatsphäre $\to$ Entscheidung zwischen beiden notwendig $\to$ Bedenken beim Data Sharing, ggf. zurückhaltend
    \item mangelnde Kooperation / Vertrauen $\to$ mehrfache Speicherung derselben Daten bei verschiedenen Unternehmen $\to$ inkonsistente, veraltete Daten
    \item[$\Rightarrow$] V: Verfügbarkeit von aktuellen, konsistenten Daten
    \item[$\Rightarrow$] V: Datenschutz und Datensouveränität zusammen
\end{itemize}

\hl{aufgrund der vielen Blockierer (\emph{Strategic Disabler}) schaffen wir es aktuell nicht, Daten-getriebene Prozesse in der Breite zu etablieren}

% !TeX root = ../presentation.tex

\section{Potenziale und Ziele}

\begin{frame}[allowframebreaks]{Vertrauen}
    \textbf{Warum kein Vertrauen?}
    \begin{itemize}
        \item Angst vor Datenmissbrauch oder unberechtigter Weitergabe~\cite{mollerIndustrialDataEcosystems2024}
    \end{itemize}

    \textbf{Was würde helfen?}
    \begin{itemize}
        \item Verwaltung von Daten durch Nutzende unabhängig von Anwendungen
        \item Daten im Besitz der Nutzenden
        \item Kontrolle über Daten und Zugang
        \item[$\Rightarrow$] Datensouveränität \& Datenschutz
    \end{itemize}

    \framebreak

    \textbf{Wem kann ich vertrauen? Ist das wirklich diese Person?}
    \begin{itemize}
        \item[$\to$] Authentifizierungsverfahren
        \item[$\to$] Vertrauen in geteilte Daten
        \item[$\Rightarrow$] basierend auf Netzwerk $\to$ \emph{Web of Trust}
        \item[$\Rightarrow$] ad"=hoc Data Sharing möglich
    \end{itemize}

    \framebreak

    \textbf{Einhaltung gesetzlicher Maßnahmen}
    \begin{itemize}
        \item technische Maßnahmen notwendig
        
        \item physische und logische Trennung von Datenspeichern je nach Kontext
        \begin{itemize}
            \item bspw. Medizin"= getrennt von Bewerbungsdaten
        \end{itemize}

        \item[$\Rightarrow$] Dezentralisierung
        \item[$\Rightarrow$] Interoperabilität auf Daten"= statt Anwendungsebene
    \end{itemize}
\end{frame}


\begin{frame}{Aktualität und Verfügbarkeit von Daten}
    \textbf{Warum veraltet und schlecht verfügbar?}
    \begin{itemize}
        \item Angst vor Datenmissbrauch und unberechtigter Weitergabe $\to$ mangelndes Vertrauen
        \item[$\to$] Datensouveränität und Datenschutz
        \item[$\to$] höhere Wahrscheinlichkeit zur Speicherung von mehr und diverseren Daten
        \item[$\Rightarrow$] Aktualität und Verfügbarkeit von Daten
    \end{itemize}
\end{frame}


\begin{frame}{Kosten, Effizienz und Geschwindigkeit}
    \textbf{Was ist so teuer?}
    \begin{itemize}
        \item aufwendige Integrationen durch unterschiedliche Datenstrukturen
        \item wiederholte Schritte, bspw. Implementierung von Standardfunktionalitäten
    \end{itemize}

    \textbf{Was würde helfen?}
    \begin{itemize}
        \item Auslagerung von Standardfunktionalitäten, Wiederverwendbarkeit, Interoperabilität
        \item[$\Rightarrow$] Automatisierung $\Rightarrow$ Kostensenkung, schnelle Prozesse
        \item[$\Rightarrow$] ad-hoc Zusammenschaltung von Geschäftsprozessen
    \end{itemize}
\end{frame}


\begin{frame}{Zugänglichkeit}
    \begin{itemize}
        \item[?] hohe Kosten und Aufwand
        \item[?] wenig verfügbare Daten, proprietäre Datenformate
    \end{itemize}

    \begin{itemize}
        \item Automatisierung $\to$ niedrigere Kosten und Aufwand
        \item breitere Verfügbarkeit von Daten, Interoperabilität
        \item[$\Rightarrow$] niedrigere Einstiegsbarrieren
        \item[$\Rightarrow$] Zugänglichkeit $\Rightarrow$ Wettbewerb, Innovation
    \end{itemize}
\end{frame}


\begin{frame}{Wie erreichen wir das?}
    \Large Kooperation verschiedener Akteure in einem vertrauenswürdigen, offenen, interoperablen System für Data Sharing
\end{frame}
% !TeX root = ../presentation.tex

\section{Data Spaces}

\begin{frame}{Inter-Organisational Information Systems (IOIS) \footnotesize\cite{mollerIndustrialDataEcosystems2024}}
    \begin{columns}
        \begin{column}{0.6\textwidth}
            \begin{itemize}
                \item \alert{bilaterale} Beziehungen, bspw. Lieferketten
                \item tiefe Integration, automatisiertes Data Sharing
                \item zweckgebunden, bspw. Koordinierung und Optimierung von Lieferketten
                % \item Mittel zum Zweck
                
                \item<2-> mangelndes Vertrauen, streng formalisierte Nutzungsrichtlinien
                \item<2-> keine Daten teilen? $\to$ Ineffizienz $\to$ Verlust
                \item<2-> Herausforderungen bei Skalierung % Datenaustausch, Kontrolle
            \end{itemize}
        \end{column}
        
        \begin{column}{0.4\textwidth}
            \begin{figure}
                \includegraphics[height=0.5\textheight]{./assets/iois_architecture.drawio.pdf}
                \caption{IOIS Architektur}
            \end{figure}
        \end{column}
    \end{columns}
\end{frame}


\begin{frame}{Data Intermediaries \footnotesize\cite{mollerIndustrialDataEcosystems2024}}
    \begin{columns}
        \begin{column}{0.6\textwidth}
            \begin{itemize}
                % \item dynamisch, geteilter Zweck
                \item Gleichgewicht zwischen erhaltenem und gegebenem Aufwand
                \item Daten als strategische Ressource % statt Mittel zum Zweck
                % \item ermöglichen neue Geschäfte und Optimierung von Prozessen
                
                \item<2-> Interaktion und Kooperation \alert{multilateraler} Akteure
                \item<2-> Data User, Data Provider, Data Intermediary
                
                \item<3-> offener, dynamischer Datenaustausch
                \item<3-> Netzwerkeffekte, Förderung von Innovation
            \end{itemize}
        \end{column}
        
        \begin{column}{0.4\textwidth}
            \begin{figure}
                \centering
                \includegraphics[height=0.5\textheight]{./assets/industrial_de_architecture.drawio.pdf}
                \caption{Industrial Data Ecosystems mit Data Intermediary}
            \end{figure}
        \end{column}
    \end{columns}
\end{frame}


\begin{frame}{Data Spaces \footnotesize\cite{mollerIndustrialDataEcosystems2024}}
    \begin{columns}
        \begin{column}{0.6\textwidth}
            \begin{itemize}
                \item Vereinen von IOIS und Data Intermediaries
                \item dezentrale Speicherung von Daten bei Provider
                
                % TODO: ???
                \item<2-> \emph{Data Space Connectors} für bilateralen Datenaustausch % vgl. IOIS
                \item<2-> Zusammenbringen von Data User und Provider % vgl. Data Intermediary
                
                \item<3-> technische Garantie von Datensouveränität
                % -> Überwindung betriebl. Barrieren
                % Kontrolle über Zugriff und Verwendung bei Provider
                \item<3-> geteilter Raum für vertrauenswürdiges Data Sharing $\to$ Optimierung, Innovation
            \end{itemize}
        \end{column}
        
        \begin{column}{0.4\textwidth}
            \begin{figure}
                \includegraphics[height=0.5\textheight]{./assets/data_space_architecture.drawio.pdf}
                \caption{Data Space Architektur}
            \end{figure}
        \end{column}
    \end{columns}
\end{frame}


\begin{frame}[c]{Data Spaces II \footnotesize\cite{mollerIndustrialDataEcosystems2024}}
    % verteilt by Design --> Daten bleiben bei Quelle, d.h. Data Provider
    % Zugang nur gewährt, wenn notwendig
    \vspace{1.5em}
    \begin{figure}
        \includegraphics[height=0.6\textheight]{./assets/central_vs_decentral.drawio.pdf}
        \caption{Symbolbild: zentralisierte vs. dezentralisierte Datenspeicherung}
    \end{figure}
\end{frame}


\begin{frame}{Data Spaces III \footnotesize\cite{mollerIndustrialDataEcosystems2024}}
    \begin{columns}
        \begin{column}{0.6\textwidth}
            \begin{itemize}
                \item dezentralisierte Datenspeicherung
                \begin{itemize}
                    \item[$\to$] verteilt \emph{by Design} %, Datenredundanz % TODO: Datenredundanz?
                \end{itemize}
                
                \item<2-> Datenintegration auf semantische Ebene
                \begin{itemize}
                    \item[$\to$]<2-> kein einheitliches Daten"=Schema notwendig
                \end{itemize}
                
                \item<3-> Verschachtelung / Überlappung von Data Spaces
                \begin{itemize}
                    \item[$\to$]<3-> \emph{Data Ecosystems}
                    % um einen oder mehrere föderierte Data Spaces
                    % technische Integration über Schnittstellen
                    % Datensouveränität & Verhinderung großer Daten-Silos: verschachtelt && überlappend
                \end{itemize}
                
                \item<4-> Integration über Schnittstellen
                \item<4-> Erreichen gemeinsamer Ziele
            \end{itemize}
        \end{column}

        \begin{column}{0.4\textwidth}
            \only<3->{
                \begin{figure}
                    \includegraphics[height=0.5\textheight]{./assets/data_ecosystem_architecture.drawio.pdf}
                    \caption{Data Ecosystem}
                \end{figure}
            }
        \end{column}
    \end{columns}
\end{frame}


\begin{frame}{Data Spaces IV \footnotesize\cite{mollerIndustrialDataEcosystems2024}}
    \begin{itemize}
        \item flexible betriebliche Strukturen
        \item Zugang nur für bestimmte Akteure $\to$ \emph{Trusted Pool}
        \item sicheres, vertrauenswürdiges Data Sharing
        
        \pause
        \item Einbettung in Data Ecosystem
        \item \emph{Data Space Member} vs. \emph{Data Ecosystem Party}
        % DSM: direkt technisch in DS eingebunden
        % DEP: nur indirekter Zugriff über Data Space Connector (Schnittstelle zu DSM)
        
        \pause
        \item Governance"=Maßnahmen auf allen Abstraktionsebenen
        % Erfüllung rechtlicher Rahmenbedingungen
        % Ebene von DE, DS oder Use Case
    \end{itemize}
    
    % Data Spaces erfüllen am meisten Kriterien
    % Wie kann das funktionieren? --> Social Linked Data
\end{frame}

% !TeX root = ../presentation.tex

\section{Social Linked Data}

\begin{frame}{Social Linked Data \footnotesize\cite{mecklerWebLinkedData2023}}
    \begin{columns}
        \begin{column}{0.6\textwidth}
            \begin{itemize}
                % mögliche Implementierung des Data Space Konzeptes basiert auf Solid
                \item Data Space Konzept basierend auf\\
                      \emph{Social Linked Data} (Solid)
                \item Ziel: offene, dezentralisierte Netzwerke für souveränen Datenaustausch
                \item Definition von Protokollen für Verwaltung und Austausch von Daten, Zugriffskontrolle und Identitätsmanagement
            \end{itemize}
        \end{column}

        \begin{column}{0.4\textwidth}
            \begin{figure}
                \includegraphics[width=0.5\textwidth]{./assets/solid_logo.pdf}
                \caption{Solid-Logo~\cite{solidcommunitygroupSolidemblemsvg2019}}
            \end{figure}
        \end{column}
    \end{columns}
\end{frame}


\begin{frame}{Social Linked Data II \footnotesize\cite{mecklerWebLinkedData2023}}
    \vspace{1em}
    \begin{figure}
        \includegraphics[height=5cm]{./assets/solid_triangle.drawio.pdf}
        \caption{Solid-Komponenten}
    \end{figure}
    % - Gliederung von Anwendungen in drei Teile
    %   - Anwendung als solches // Daten // Identität
    % - Identitätskomp.: verifiziert Identität eines Akteurs zur Auth. für Datenzugriff
    % - Daten und zugehörige Zugriffsregeln: dezentral in einen oder mehreren *Personal Online Data Stores* (Pods) gespeichert
    % - Anwendung verwendet ID, um korrekten Pod zu identifizieren und um sich für Datenzugriff zu auth.
    % - anschließend kann Anw. erforderliche Daten aus Pods lesen/schreiben
\end{frame}


\begin{frame}{Social Linked Data III \footnotesize\cite{mecklerWebLinkedData2023}}
    \begin{columns}
        \begin{column}{0.4\textwidth}
            \begin{itemize}
                \item Trennung und Standardisierung
                \item[$\Rightarrow$] Austauschbarkeit von Komponenten
                \item[$\Rightarrow$] Schritt"=für"=Schritt"=Einführung
                \item[$\Rightarrow$] geringe Einstiegsbarriere
                \item[$\Rightarrow$] hohe Zugänglichkeit
            \end{itemize}
        \end{column}

        \begin{column}{0.6\textwidth}
            \begin{figure}
                \includegraphics[width=\textwidth]{./assets/solid_triangle.drawio.pdf}
            \end{figure}
        \end{column}
    \end{columns}
\end{frame}


\subsection{Datenmanagement und Datenzugriff}

\begin{frame}{Datenstruktur}
    \begin{itemize}
        \item dezentrale Speicherung in \emph{Personal Online Data Stores} (Pods)~\cite{mecklerWebLinkedData2023,sambraSolidPlatformDecentralized2016}
        \item Standardisierung $\to$ Interoperabilität auf Daten"= statt Anwendungsebene
              % --> einfacher Wechsel von Anwendungen ohne aufwendige Datenmigration
        
        \item Binärdaten\only<1>{?} \only<2->{mit Metadaten}
        
        \pause
        \item lesbares Format \only<3->{$\to$ \emph{Resource Description Framework} (RDF) mit \emph{Vocabularies}~\cite{mecklerWebLinkedData2023,sambraSolidPlatformDecentralized2016}}
        % bspw. <Tim Berners-Lee> <is a> <person> (vgl. Bizer)
        
        \pause
        \pause
        \item global eindeutige Identifikation via \emph{Uniform Resource Identifier} (URI)~\cite{sambraSolidPlatformDecentralized2016}
        \begin{itemize}
            \item \texttt{https://www.w3.org/People/Berners-Lee/card\#i}~\cite{bizerLinkedDataStory2009}
        \end{itemize}

        \pause
        \item Verknüpfung via \emph{Linked Data} $\to$ Struktur \& automatisierbare Semantik~\cite{bizerLinkedDataStory2009,mecklerWebLinkedData2023}
        
        \pause
        \item Mapping anderer Strukturen zu RDF~\cite{mecklerWebLinkedData2023,sambraSolidPlatformDecentralized2016}
    \end{itemize}
\end{frame}

% TODO: RDF Erklärung? --> Bizer

\begin{frame}{Datenstruktur II \footnotesize\cite{mecklerWebLinkedData2023,sambraSolidPlatformDecentralized2016}}
    \begin{columns}
        \begin{column}{0.6\textwidth}
            \begin{itemize}
                \item (nicht-) RDF"=Dateien in \emph{LDP"=Container}
                \item wiederum RDF"=Graph $\to$ Verschachtelung möglich
                \item Zugriffskontrolle auf jeder Ebene mittels \emph{Access Control List} (ACL)
                \begin{itemize}
                    \item \texttt{resource.acl} oder \texttt{.acl}
                \end{itemize}
                
                \item<2> unterschiedliche Rechte pro Akteur / Container
                \item[$\Rightarrow$]<2> feingranularer Datenschutz und Zugriffskontrolle
            \end{itemize}
        \end{column}

        \begin{column}{0.4\textwidth}
            \vspace{1em}
            \begin{figure}
                \includegraphics[height=4.5cm]{./assets/container_hierarchy.drawio.pdf}
                \caption{Hierarchie \cite[vgl.][]{sambraSolidPlatformDecentralized2016}}
            \end{figure}
        \end{column}
    \end{columns}
\end{frame}


\begin{frame}{Read- und Write-Protokoll \footnotesize\cite{mecklerWebLinkedData2023,sambraSolidPlatformDecentralized2016}}
    \begin{itemize}
        \item Anwendungen lesen und schreiben Daten direkt aus Pods
        \item Interoperabilität der Pods mit Anwendungen\\
              $\to$ wohldefiniertes, einfach implementierbares Protokoll\\
              \only<2->{$\Rightarrow$ RESTful"=Methoden basierend auf \emph{Linked Data Platform} (LDP)}
        \item<3-> komplizierte Datenabfragen mittels SPARQL (optional)
    \end{itemize}
\end{frame}


% \begin{frame}{Datenabfragen \footnotesize\cite{sambraSolidPlatformDecentralized2016}}
%     \begin{itemize}
%         \item nur einfache Abfragen mittels LDP"=Methoden möglich
%         \item komplizierte Datenabfragen mittels SPARQL (optional)
%         \item Delegation an Server $\Rightarrow$ Entwicklungsaufwand $\downarrow$
        
%         \pause
%         \item \emph{Local Queries}: innerhalb \emph{eines} Pods
%         \item \emph{Link Following Queries}: über mehrere Pods hinweg
%         \begin{itemize}
%             \item via Link"=Following
%             \item tatsächliche Verteilung muss nicht bekannt sein
%         \end{itemize}
%     \end{itemize}
% \end{frame}


\subsection{Identität und Authentifizierung}

\begin{frame}{Authentifizierung \footnotesize\cite{sambraSolidPlatformDecentralized2016}}
    \begin{itemize}
        \item Vertrauen, Datensouveränität und Datenschutz $\to$ Authentifizierung
        \item Dezentralisierung benötigt globalen \emph{Identity Space}
        
        \pause
        \item passend zu RDF"=basierten Daten
        \item Ermittlung der Identität und Profildaten
        \item Ermittlung relevanter Links zum Pod und zu Anwendungsdaten
        
        \pause
        \item[$\Rightarrow$] \emph{WebID}
        \item[$\Rightarrow$] globales Identitätsmanagement basierend auf System dezentralisierter \emph{Identity Provider}
    \end{itemize}
\end{frame}


\begin{frame}{Identität}
    \begin{itemize}
        \item Akteure besitzen WebID"=URI~\cite{sambraSolidPlatformDecentralized2016}
        
        \item Referenz auf \emph{WebID Profile Document}~\cite{sambraSolidPlatformDecentralized2016,solidcommunitygroupSolidemblemsvg2019}
        \begin{itemize}
            \item Referenz auf Pod und Anwendungsdaten~\cite{solidcommunitygroupSolidWebIDProfile2024}
            \item Referenz auf weitere Profildaten~\cite{solidcommunitygroupSolidWebIDProfile2024}
            \item Webseite im RDF"=Format, global eindeutige URI~\cite{sambraSolidPlatformDecentralized2016}
        \end{itemize}
        
        \item Speicherung bei Identity Provider (meist Pod Provider)~\cite{sambraSolidPlatformDecentralized2016}
        \item[$\Rightarrow$] Kontrolle über eigene Identität bei Nutzenden~\cite{sambraSolidPlatformDecentralized2016}
    \end{itemize}
\end{frame}


\begin{frame}{Web of Trust \footnotesize\cite{sambraSolidPlatformDecentralized2016}}
    \begin{columns}
        \begin{column}{0.6\textwidth}
            \begin{itemize}
                \item Verknüpfung von Identitäten über mehrere Seiten\\
                      $\Rightarrow$ \emph{Web of Trust}
                
                \item[$\Rightarrow$]<2-> ad-hoc Auth.-Entscheidungen basierend auf Profileigenschaften
                    % bspw. Beziehungen zu anderen Agenten, Arbeitsstelle, Teil einer Gruppe etc. etc.
                
                \item[$\Rightarrow$]<3-> Wem kann ich vertrauen?\\ Kann ich den geteilten Daten vertrauen?
                
                \item[$\Rightarrow$]<4-> Adressierung eines Kern"=Hindernisses
            \end{itemize}
        \end{column}

        \begin{column}{0.4\textwidth}
            \vspace{1em}
            \begin{figure}
                \includegraphics[height=4cm]{./assets/web_of_trust.drawio.pdf}
                \caption{Web of Trust}
            \end{figure}
        \end{column}
    \end{columns}
\end{frame}


\subsection{Erweiterung: B2B-Wertschöpfungsketten}

\begin{frame}{B2B-Wertschöpfungsketten \footnotesize\cite{bothSolidBasedB2BData2025}}
    \begin{itemize}
        \item automatisierte Datenübertragung essenziell für Partner in Wertschöpfungsketten
        \item weitere Anforderungen
        \begin{itemize}
            \item garantierte, nachvollziehbare Einhaltung rechtlicher Rahmenbedingungen
            \item Constraints für Data Sharing
        \end{itemize}
    \end{itemize}

    \pause
    \begin{itemize}
        \item Einführung zusätzlicher Metadaten
        \begin{itemize}
            \item \emph{Data Processing Purpose}
            \item Verstecken der Datenquelle, Angabe über Weitergabe
        \end{itemize}
    \end{itemize}

    \pause
    \begin{itemize}
        \item weitere Validierung notwendig
        % \item Möglichkeit und Notwendigkeit zur Erweiterung von Solid
        % \item weiterer Schritt in Richtung E2E-B2B-Wertschöpfungsketten
    \end{itemize}
    % TODO: Stichpunkte zum Erzählen
\end{frame}

% !TeX root = ../presentation.tex

\section{Einschätzung und Zukunft}

\begin{frame}{Einschätzung und Zukunft}
    TBA
\end{frame}

% !TeX root = ../handout.tex

\section{Fazit}

\begin{itemize}
    % \item Vision: vertrauenswürdiges Data Sharing, aktuelle und konsistente Daten, Zugänglichkeit, effiziente Integrationen, Förderung von Innovation und Kooperation

    \item Data Spaces sind dezentrale, multilaterale Informationssysteme, ermöglichen vertrauenswürdiges Data Sharing unter Datensouveränität

    \item Social Linked Data ist offener Standard zum Verwalten und Teilen von Daten im Web, Verwendung zum Erschaffen von Data Spaces
    
    \item Dezentrale Datenspeicherung in Pods, volle Kontrolle durch Nutzende, Wechsel von Datenspeicher unabhängig von Anwendung
    
    % Daten werden dezentral in Pods gespeichert, wobei Nutzende die volle Kontrolle über den Zugang haben.
    %       Durch Standardisierung und Interoperabilität können Datenspeicher unabhängig von Anwendungen gewechselt werden.
    
    % \item Daten sind in RDF strukturiert, sodass diese automatisierbare Semantik enthalten, und verwandten Ressourcen verlinkt sind (Linked Data).
    
    % \item Zukunft: Digitalisierung, datensouveränes Data Sharing, Mapping statt ETL, Data  Mesh statt Data Lake, Metaprogrammierung mit agnostischen Daten, Nutzerzentrierung von Anwendungen
\end{itemize}


% !TeX root = presentation.tex

{
    \metroset{sectionpage=none}
    \section{Literatur}
    \begin{frame}[allowframebreaks]{Literatur}
        \printbibliography
    \end{frame}
}

% !TeX root = presentation.tex

\begin{frame}{License}
    Diese Folien sind unter der \href{https://creativecommons.org/licenses/by/4.0/}{Creative Commons Attribution 4.0 International} lizenziert. \ccby
    
    Der Quellcode kann unter folgendem Link abgerufen werden:

    \url{\repourl}
\end{frame}


\begin{frame}{Fazit}
    \begin{itemize}
        \item \textbf{Vision}: vertrauenswürdiges Data Sharing, aktuelle und konsistente Daten, Zugänglichkeit, Effizienz, Innovation und Kooperation
        \item \textbf{Data Spaces}: dezentrale, multilaterale Informationssysteme; ermöglichen vertrauenswürdiges Data Sharing unter Datensouveränität
        \item \textbf{Solid}: offener Standard zum Verwalten und Teilen von Daten im Web
        \begin{itemize}
            \item dezentrale Datenspeicherung in Pods; RDF und WebID; volle Kontrolle durch Nutzende; Wechsel von Datenspeicher unabhängig von Anwendung
        \end{itemize}
        \item \textbf{Zukunft}: Dezentralisierung, Mapping statt ETL, Metaprogrammierung mit agnostischen Daten, Nutzerzentrierung von Anwendungen
    \end{itemize}

    \note{
        \textbf{Diskussion}
        \begin{itemize}
            \item Was haltet ihr von dem Ansatz?
            \item Würdet ihr so einem Ansatz vertrauen?
            \item Wie kann man die Big Player dazu bewegen, so ein Konzept zu adaptieren?
        \end{itemize}
    }
\end{frame}

\end{document}
