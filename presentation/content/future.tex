% !TeX root = ../presentation.tex

\section{Einschätzung und Zukunft}

\begin{frame}{Einschätzung}
    \begin{itemize}
        \item Zusammenführung vieler Ziele, um Vision näher an Realität zu bringen
        
        \pause
        \item Vertrauen?
        \pause
        \begin{itemize}
            \item Ansatz für Datensouveränität, Datensilos? gesetzliche Vorgaben?
        \end{itemize}
        % Ansatz zur Gewährleistung von Datensouveränität
        % unterschiedliche Datenspeicher je nach Kontext, Web of Trust
        % größerer Anreiz für Speicherung von Daten (<-- mehr Vertrauen und Kontrolle)
        % Web of Trust: ein kompromittierter Akteur reicht, um Schaden anzurichten
        % Startpunkt, aber wie verhindern, dass Data User die Daten zwischenspeichern?
        % Einhaltung gesetzlicher Vorgaben? --> o.g. Ansatz (Metadaten) benötigt weitere Forschung

        \pause
        \item Aktualität und Konsistenz von Daten?
        \pause
        \begin{itemize}
            \item Dezentralisierung und Datensouveränität, Anreiz $\uparrow$, Standardisierung
        \end{itemize}
        % mehr Vertrauen + Dezentralisierung + Datensouveränität 
        %     --> Daten bei Nutzenden
        %     --> größerer Anreiz zum Speichern von Daten
        % können diese einfach aktuell halten
        %     im Vgl. zu zentralisierten Ansatz, wo dieselben Daten verteilt bei vielen Unternehmen liegt
        % Standardisierung --> größere Verfügbarkeit von Daten

        \pause
        \item Effizienz und Geschwindigkeit?
        \pause
        \begin{itemize}
            \item Interoperabilität, automatisierbare Semantik, ggf. Mapping
        \end{itemize}
        % standardisierte Speicherung von Daten
        % Interoperabilität auf Daten- statt Anwendungsebene
        % automatisierbare Semantik durch RDF + Vocabularies
        % dadurch schnellere Datenintegration möglich, ggf. Mapping notwendig (teil- / automatisiert)

        \pause
        \item Zugänglichkeit unabhängig von Größe, Branche etc.?
        \pause
        \begin{itemize}
            \item Interoperabilität, Aufwand $\downarrow$, niedrigere Einstiegsbarriere

            \pause
            \item[$\Rightarrow$] Förderung von Kooperation und Innovation
        \end{itemize}
        % Interoperabilität auf Daten- statt Anw.-Ebene
        %     --> keine eigenen Daten mehr notwendig
        %     --> Zugänglichkeit von Daten
        % verminderter Aufwand durch Auslagerung von Std.-Fkt. + Wiederverwendbarkeit
        % niedrigere Einstiegsbarriere
        % Zugänglichkeit des Marktes
    \end{itemize}
\end{frame}


\begin{frame}{Zukunft}
    % TODO: Beispiel durchziehen
    % stellen uns vor: cooles neues Projekt
    % gehe zu neuen potenziellen Gesch.-Partnern und bringe meine Akten-Tasche voller Daten mit
    % ich kann alle meine Daten immer mit mir führen

    % dafür müssen sie digital sein

    \begin{itemize}
        \item Digitalisierung zum Profitieren von Vorteilen
        % digitales Bild nicht ausreichend, brauchen Daten mit automatisierbarer Semantik

        % interoperabel zur einfachen Zusammenführung
        \pause
        \item Dezentralisierung und Interoperabilität
        \begin{itemize}
            \item[$\to$] Fokus auf dezentrale Architekturen
        \end{itemize}

        % damit ermögliche ich ...
        \pause
        \item Ermöglichung von ad"=hoc Daten- und Anwendungsintegrationen
        \begin{itemize}
            \item[$\to$] Mapping statt aufwendige ETL"=Prozesse
                % Wiederverwendung, Schnittstellen
            
            % dyn. Interpr. durch Semantik aus RDF-Daten -->
            \item[$\to$] Abstraktion von Domänen-Daten, Metaprogrammierung
            % Branchen-agnostische Daten
            % Zukunft: Agenten interpretieren Daten nach meinen Kriterien, unabh. ob es User, Patient oder Benutzer heißt
        \end{itemize}

        % wird begünstigt durch
        \pause
        \item Datenstrukturen unabhängig von Anwendungen, Verfügbarkeit von Daten
        \begin{itemize}
            % aktuell: Wenn ich das mache, gehen Kunden weg?
            %          Einsperren von Kunden durch proprietäre Datenformate
            \item[$\to$] neue Art und Weise der Unterscheidung auf Markt
            \item[$\to$] Nutzerzentrierung, steigender Qualitätsanspruch
            % gutes Produkt zieht Kund. an, schlechtes stößt ab
            % begünstigt durch niedrigere Einstiegshürden in Markt
            % TODO: Blocker: Big Player werden dagegen wirken
        \end{itemize}

        \pause
        \item[$\Rightarrow$] Daten-orientiertes Vorgehen statt Prozess-Orientierung
        % Daten als Antreiber statt Prozesse
    \end{itemize}
\end{frame}
