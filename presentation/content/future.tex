% !TeX root = ../presentation.tex

\section{Einschätzung und Zukunft}

\begin{frame}{Einschätzung}
    \begin{itemize}
        \item Zusammenführung vieler Ziele, um Vision näher an Realität zu bringen
        
        \pause
        \item Vertrauen?
        \pause
        \begin{itemize}
            \item Ansatz für Datensouveränität, Datensilos? gesetzliche Vorgaben?
        \end{itemize}

        \note{
            \scriptsize
            \begin{itemize}
                \item Ansatz zur Gewährleistung von Datensouveränität, untersch. Datenspeicher je nach Kontext, Web of Trust\\
                    größerer Anreiz für Speicherung von Daten ($\leftarrow$ mehr Vertrauen und Kontrolle)\\
                    Web of Trust: ein kompromittierter Akteur reicht, um Schaden anzurichten\\
                    Startpunkt, aber wie verhindern, dass Data User die Daten zwischenspeichern?\\
                    Einhaltung gesetzlicher Vorgaben? $\to$ o.g. Ansatz (Metadaten) benötigt weitere Forschung
            \end{itemize}
        }

        \pause
        \item Aktualität und Konsistenz von Daten?
        \pause
        \begin{itemize}
            \item Dezentralisierung und Datensouveränität, Anreiz $\uparrow$, Standardisierung
        \end{itemize}

        \note{
            \scriptsize
            \begin{itemize}
                \item mehr Vertrauen + Dezentralisierung + Datensouveränität \\
                    $\to$ Daten bei Nutzenden $\to$ größerer Anreiz zum Speichern von Daten\\
                    $\to$ aktuell halten im Vgl. zentral. Ansatz (gleiche Daten bei viele Untern.)\\
                    Standardisierung $\to$ größere Verfügbarkeit von Daten
            \end{itemize}
        }

        \pause
        \item Effizienz und Geschwindigkeit?
        \pause
        \begin{itemize}
            \item Interoperabilität, automatisierbare Semantik, ggf. Mapping
        \end{itemize}

        \note{
            \scriptsize
            \begin{itemize}
                \item standardisierte Speicherung von Daten, Interoperabilität auf Daten- statt Anw.-Ebene\\
                    autom. Sem. durch RDF + Vocabularies $\to$ schnellere Daten-Int. möglich, ggf. Mapping (teil- / automatisiert)
            \end{itemize}
        }

        \pause
        \item Zugänglichkeit unabhängig von Größe, Branche etc.?
        \pause
        \begin{itemize}
            \item Interoperabilität, Aufwand $\downarrow$, niedrigere Einstiegsbarriere

            \pause
            \item[$\Rightarrow$] Förderung von Kooperation und Innovation
        \end{itemize}

        \note{
            \scriptsize
            \begin{itemize}
                \item Interop. Daten- statt Anw.-Eb. $\to$ keine eigenen Daten mehr notw. $\to$ Zugänglichkeit v. Daten\\
                verminderter Aufwand durch Auslagerung von Std.-Fkt. + Wiederverwendbarkeit $\to$ niedrigere Einstiegsbarriere\\
                Zugänglichkeit des Marktes
            \end{itemize}
        }
    \end{itemize}
\end{frame}


\begin{frame}{Zukunft}
    % dafür müssen sie digital sein

    \begin{itemize}
        \item Digitalisierung zum Profitieren von Vorteilen
        
        \note{
            \scriptsize
            \begin{itemize}
                \item digitales Bild nicht ausreichend, brauchen Daten mit automatisierbarer Semantik\\
                    interoperabel zur einfachen Zusammenführung
            \end{itemize}
        }

        \pause
        \item Dezentralisierung und Interoperabilität
        \begin{itemize}
            \item[$\to$] Fokus auf dezentrale Architekturen
        \end{itemize}

        \pause
        \item Ermöglichung von ad"=hoc Daten- und Anwendungsintegrationen
        \begin{itemize}
            \item[$\to$] Mapping statt aufwendige ETL"=Prozesse
            \item[$\to$] Abstraktion von Domänen-Daten, Metaprogrammierung
        \end{itemize}

        \note{
            \scriptsize
            \begin{itemize}
                \item Mapping, Wiederverwendung, Schnittstellen\\
                    dyn. Interpr. durch Semantik aus RDF-Daten $\to$ Abstraktion\\
                    Branchen-agnostische Daten\\
                    Zukunft: Agenten interpretieren Daten nach meinen Kriterien, unabh. ob es User, Patient oder Benutzer heißt
            \end{itemize}
        }

        % wird begünstigt durch
        \pause
        \item Datenstrukturen unabhängig von Anwendungen, Verfügbarkeit von Daten
        \begin{itemize}
            \item[$\to$] neue Art und Weise der Unterscheidung auf Markt
            \item[$\to$] Nutzerzentrierung, steigender Qualitätsanspruch
        \end{itemize}

        \note{
            \scriptsize
            \begin{itemize}
                \item aktuell: gehen Kunden weg, wenn ich das mache? Einsperren durch prop. Formate\\
                    gutes Produkt zieht Kund. an, schlechtes stößt ab\\
                    begünstigt durch niedrigere Einstiegshürden in Markt\\
                    Blocker: Big Player werden dagegen wirken
            \end{itemize}
        }

        \pause
        \item[$\Rightarrow$] Daten-orientiertes Vorgehen statt Prozess-Orientierung

        \note{
        \begin{itemize}
            \item Daten als Antreiber statt Prozesse\\
                Vorstellung: kann mit einer innovativen Idee direkt zu potenz. Geschäftspartner gehen, digitale Akten"=Tasche
        \end{itemize}
    }
    \end{itemize}
\end{frame}
