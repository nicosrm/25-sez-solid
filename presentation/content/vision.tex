% !TeX root = ../presentation.tex

\section{Vision}

\begin{frame}{Strategische Ressource {\footnotesize\cite{mollerIndustrialDataEcosystems2024}}}
    \begin{itemize}
        \item Daten: vom \enquote{Nebenprodukt} zur strategischen Ressource
        \item Datenaustausch essenziell für Geschäftsprozesse
        \item Koordination von Lieferketten, Erfüllung rechtlicher Rahmenbedingungen
        
        \item viele Bedenken beim Data Sharing
        \begin{itemize}
            \item Missbrauch von Daten
            \item Angst vor unberechtigter Weitergabe (Geschäftsgeheimnisse)
        \end{itemize}

        \item meist nur bilateraler, eingeschränkter Datenaustausch

        \item[$\Rightarrow$] vertrauensvolles, multilaterales Data Sharing
    \end{itemize}
\end{frame}


\begin{frame}{Aufwendige Datenintegration}
    \begin{itemize}
        \item Datenintegration $\to$ Optimierung von Geschäftsprozessen
        \item Integration mehrerer Projekte oft mittels \emph{Extract, Transform, Load}
        \item wiederholte Schritte, zeitintensiv, teuer, inkonsistent
        \item hohe Einstiegsbarriere für neue Akteure
        
        \item[$\Rightarrow$] schnelles, effizientes, günstiges Verfahren
        \item[$\Rightarrow$] Datenkonsistenz
        \item[$\Rightarrow$] Begünstigung innovativer Lösungen
    \end{itemize}
\end{frame}


\begin{frame}{Datensilos und Datensouveränität}
    \begin{itemize}
        \item Daten als wertvolles Wirtschaftsgut $\to$ Datensilos
        \item Kontrolle \emph{oder} Privatsphäre $\to$ zurückhaltendes Data Sharing
        \item mangelnde Kooperation $\to$ mehrfache Speicherung derselben Daten\\
              $\to$ veraltete, inkonsistente Daten
        
        \item[$\Rightarrow$] Verfügbarkeit von aktuellen, konsistenten Daten
        \item[$\Rightarrow$] Datenschutz \emph{und} Datensouveränität
    \end{itemize}
\end{frame}


\begin{frame}{Folgen}
    \begin{itemize}
        \item Daten"=getriebene Prozesse noch nicht in Breite etabliert
        \item kein vollständiges Ausnutzen des Potenzials für Wirtschaft

        \item überteuerte, langwierige Projekte
        \begin{itemize}
            \item bspw. Flughafen Berlin Brandenburg (2,8x Bauzeit, 3x Kosten)~\cite{stalinskiBestBERZahlen2020}
        \end{itemize}

        \item \enquote{We've lost control of our personal data} -- Tim Berners"=Lee~\cite{berners-leeThreeChallengesWeb2017}
    \end{itemize}
\end{frame}
