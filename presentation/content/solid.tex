% !TeX root = ../presentation.tex

\section{Solid}

\begin{frame}{Solid \footnotesize\cite{mecklerWebLinkedData2023}}
    \begin{columns}
        \begin{column}{0.6\textwidth}
            \begin{itemize}
                \item Data Space Konzept basierend auf\\
                    \emph{Solid} (ehem. Social Linked Data)
                \item<2-> Ziel: offene, dezentralisierte Netzwerke für souveränen Datenaustausch
                \item<3-> Definition von Protokollen für Verwaltung und Austausch von Daten, Zugriffskontrolle und Identitätsmanagement
            \end{itemize}
        \end{column}
        
        \begin{column}{0.4\textwidth}
            \begin{figure}
                \includegraphics[width=0.5\textwidth]{./assets/solid_logo.pdf}
                \caption{Solid-Logo~\cite{solidcommunitygroupSolidemblemsvg2019}}
            \end{figure}
        \end{column}
    \end{columns}

    \note{
        \begin{itemize}
            \item mögliche Implementierung des Data Space Konzeptes basiert auf Solid
        \end{itemize}
    }
\end{frame}


\begin{frame}{Solid II \footnotesize\cite{mecklerWebLinkedData2023}}
    \vspace{1em}
    \begin{figure}
        \includegraphics[height=5cm]{./assets/solid_triangle.drawio.pdf}
        \caption{Solid-Komponenten}
    \end{figure}

    \note{
       \begin{itemize}
        \item Gliederung von Anwendungen in drei Teile
        \item ID-Komp.: verifiziert Identität eines Akteurs zur Auth. für Datenzugriff
        \item Daten und zugehörige Zugriffsregeln: dezentral in einen oder mehreren Datenspeichern gespeichert
        \item Anwendung verwendet ID, um korrekten Speicher zu identif. und um sich für Datenzugriff zu auth.
        \item anschließend kann Anw. erforderliche Daten aus Speicher lesen/schreiben
       \end{itemize}
    }
\end{frame}


\begin{frame}{Solid III \footnotesize\cite{mecklerWebLinkedData2023}}
    \begin{columns}
        \begin{column}{0.4\textwidth}
            \begin{itemize}
                \item Trennung und Standardisierung
                \item[$\Rightarrow$] Austauschbarkeit von Komponenten
                \item[$\Rightarrow$] Schritt"=für"=Schritt"=Einführung
                
                \item[$\Rightarrow$]<2-> geringe Einstiegsbarriere
                \item[$\Rightarrow$]<2-> hohe Zugänglichkeit
            \end{itemize}
        \end{column}
        
        \begin{column}{0.6\textwidth}
            \begin{figure}
                \includegraphics[width=\textwidth]{./assets/solid_triangle.drawio.pdf}
            \end{figure}
        \end{column}
    \end{columns}

    \note{
        \begin{itemize}
            \item an Tafel malen!
        \end{itemize}
    }
\end{frame}


\subsection{Datenmanagement und Datenzugriff}

\begin{frame}{Datenstruktur}
    \begin{itemize}
        \item dezentrale Speicherung in \emph{Personal Online Data Stores} (Pods)~\cite{mecklerWebLinkedData2023,sambraSolidPlatformDecentralized2016}
        \item Standardisierung $\to$ Interoperabilität auf Daten"= statt Anwendungsebene
        
        \note{
            \begin{itemize}
                \item einfacher Wechsel von Anwendungen ohne aufwendige Datenmigration
            \end{itemize}
        }
        
        \item \only<1|handout:0>{Binärdaten?}
              \only<2>{Binärdaten mit Metadaten}
              \only<3-|handout:0>{\st{Binärdaten mit Metadaten}}
        
        \pause
        \pause
        \item lesbares Format \only<4->{$\to$ \emph{Resource Description Framework} (RDF) mit \emph{Vocabularies}~\cite{mecklerWebLinkedData2023,sambraSolidPlatformDecentralized2016}}
        \note{
            \hrule
            \begin{itemize}
                \item \texttt{<Tim Berners-Lee> <is a> <person>}
                \item Subjekt, Prädikat, Objekt
            \end{itemize}
        }
        
        \pause
        \pause
        \item Verknüpfung via \emph{Linked Data} $\to$ Struktur \& automatisierbare Semantik~\cite{bizerLinkedDataStory2009,mecklerWebLinkedData2023}

        \pause
        \item global eindeutige Identifikation via \emph{Uniform Resource Identifier} (URI)~\cite{sambraSolidPlatformDecentralized2016}
        \begin{itemize}
            \item \texttt{https://www.w3.org/People/Berners-Lee/card\#i}~\cite{bizerLinkedDataStory2009}
            % TODO: Unterschied URL / URI
        \end{itemize}
        
        \pause
        \item Zugriffskontrolle auf jeder Hierarchie-Ebene mittels \emph{Access Control List} (ACL)
        
        \pause
        \item Mapping anderer Strukturen zu RDF~\cite{mecklerWebLinkedData2023,sambraSolidPlatformDecentralized2016}
    \end{itemize}
\end{frame}


% \begin{frame}{Datenstruktur II \footnotesize\cite{mecklerWebLinkedData2023,sambraSolidPlatformDecentralized2016}}
%     \begin{columns}
%         \begin{column}{0.6\textwidth}
%             \begin{itemize}
%                 \item (nicht-) RDF"=Dateien in \emph{LDP"=Container}
%                 \item wiederum RDF"=Graph $\to$ Verschachtelung möglich
%                 \item Zugriffskontrolle auf jeder Ebene mittels \emph{Access Control List} (ACL)
%                 \begin{itemize}
%                     \item \texttt{resource.acl} oder \texttt{.acl}
%                 \end{itemize}

%                 \item<2> unterschiedliche Rechte pro Akteur / Container
%                 \item[$\Rightarrow$]<2> feingranularer Datenschutz und Zugriffskontrolle
%             \end{itemize}
%         \end{column}

%         \begin{column}{0.4\textwidth}
%             \vspace{1em}
%             \begin{figure}
%                 \includegraphics[height=4.5cm]{./assets/container_hierarchy.drawio.pdf}
%                 \caption{Hierarchie \cite[vgl.][]{sambraSolidPlatformDecentralized2016}}
%             \end{figure}
%         \end{column}
%     \end{columns}
% \end{frame}


\begin{frame}{Read- und Write-Protokoll \footnotesize\cite{mecklerWebLinkedData2023,sambraSolidPlatformDecentralized2016}}
    \begin{itemize}
        \item Anwendungen lesen und schreiben Daten direkt aus Pods
        \item Interoperabilität der Pods mit Anwendungen\\
            und wohldefiniertes, einfach implementierbares Protokoll
        
        \note{
            \begin{itemize}
                \item möglichst viel wiederverwenden, was bereits vorhanden ist
                \item soll keine neue Komponenten sein $\to$ aufbauend auf RESTful
            \end{itemize}
        }

        \item[$\Rightarrow$]<2-> RESTful"=Service, welche \emph{Linked Data Platform} (LDP) erfüllen
        \begin{itemize}
            \item<2-> \texttt{HTTP GET}, \texttt{POST}, \texttt{PATCH}, \texttt{DELETE} etc.~\cite{sambraSolidPlatformDecentralized2016}
        \end{itemize}
        
        \note{
            \hrule
            \begin{itemize}
                \item LDP: beschreibt Datenformat
                \item Einteilung in Ressourcen und Containern, Paging
            \end{itemize}
        }

        \item<3-> komplizierte Datenabfragen mittels SPARQL (optional)
    \end{itemize}

    \note{
        \hrule\vspace{1mm}\hrule
        \begin{itemize}
            \item Fazit: Datenmanagement betrachtet
            \item dezentrale Speicherung, Interoperabilität auf Daten- statt Anw.-Ebene
            \item Wie identifizieren wir Datenspeicher und Akteure für Datenzugriff? 
        \end{itemize}
    }
\end{frame}

% \begin{frame}{Datenabfragen \footnotesize\cite{sambraSolidPlatformDecentralized2016}}
%     \begin{itemize}
%         \item nur einfache Abfragen mittels LDP"=Methoden möglich
%         \item komplizierte Datenabfragen mittels SPARQL (optional)
%         \item Delegation an Server $\Rightarrow$ Entwicklungsaufwand $\downarrow$

%         \pause
%         \item \emph{Local Queries}: innerhalb \emph{eines} Pods
%         \item \emph{Link Following Queries}: über mehrere Pods hinweg
%         \begin{itemize}
%             \item via Link"=Following
%             \item tatsächliche Verteilung muss nicht bekannt sein
%         \end{itemize}
%     \end{itemize}
% \end{frame}


\subsection{Identität und Authentifizierung}

\begin{frame}{Authentifizierung \footnotesize\cite{sambraSolidPlatformDecentralized2016}}
    \begin{itemize}
        \item Vertrauen, Datensouveränität und Datenschutz $\to$ Authentifizierung
        \item Dezentralisierung benötigt globalen \emph{Identity Space}
        
        \item<2-> passend zu RDF"=basierten Daten
        \item<2-> Ermittlung der Identität und Profildaten
        \item<2-> Ermittlung relevanter Links zum Pod und zu Anwendungsdaten
        
        \item[$\Rightarrow$]<3-> aktuell: \emph{WebID} (austauschbar)
        \item[$\Rightarrow$]<3-> globales Identitätsmanagement basierend auf System dezentralisierter \emph{Identity Provider}
    \end{itemize}
\end{frame}


\begin{frame}{Identität}
    \begin{columns}
        \begin{column}{0.55\textwidth}
            \begin{itemize}
                \item Akteure besitzen WebID"=URI~\cite{sambraSolidPlatformDecentralized2016}
                
                \item Referenz auf \emph{WebID Profile Document}~\cite{sambraSolidPlatformDecentralized2016,solidcommunitygroupSolidemblemsvg2019}
                
                \begin{itemize}
                    \item<2-> Referenz auf Pod und Anwendungsdaten~\cite{solidcommunitygroupSolidWebIDProfile2024}
                    \item<2-> Referenz auf weitere Profildaten~\cite{solidcommunitygroupSolidWebIDProfile2024}
                \end{itemize}

                \note{
                    \begin{itemize}
                        \item Webseite im RDF"=Format, global eindeutige URI~\cite{sambraSolidPlatformDecentralized2016}
                    \end{itemize}
                }
                
                \item<3-> Speicherung bei Identity Provider\\
                    (meist Pod Provider)~\cite{sambraSolidPlatformDecentralized2016}
                \item[$\Rightarrow$]<3-> Kontrolle über eigene Identität bei Nutzenden~\cite{sambraSolidPlatformDecentralized2016}
            \end{itemize}
        \end{column}

        \begin{column}{0.45\textwidth}
            \only<2->{
                \begin{figure}
                    \centering
                    \includegraphics[height=5.5cm]{./assets/profile.drawio.pdf}
                    \caption{Solid Profil~\cite[vgl.][]{sambraSolidPlatformDecentralized2016,solidcommunitygroupSolidWebIDProfile2024}}
                \end{figure}
            }
        \end{column}
    \end{columns}
\end{frame}


\begin{frame}{Web of Trust \footnotesize\cite{sambraSolidPlatformDecentralized2016}}
    \begin{columns}
        \begin{column}{0.6\textwidth}
            \begin{itemize}
                \item Verknüpfung von Identitäten über mehrere Seiten\\
                    $\Rightarrow$ \emph{Web of Trust}
                \note{
                    \begin{itemize}
                        \item hilfreich, da man nicht alle Akteure in einem DS / DE kennen kann
                    \end{itemize}
                }
                
                \item[$\Rightarrow$]<2-> ad-hoc Auth.-Entscheidungen basierend auf Profileigenschaften
                \note{
                    \hrule
                    \begin{itemize}
                        \item bspw. Beziehungen zu anderen Akteuren, Arbeitsstelle, Teil einer Gruppe etc. etc.
                        \item transitives Vertrauen
                    \end{itemize}
                }
                
                \item[$\Rightarrow$]<3-> Wem kann ich vertrauen?\\ Kann ich den geteilten Daten vertrauen?
                
                \item[$\Rightarrow$]<4-> Adressierung eines Kern"=Hindernisses
            \end{itemize}

            \note{
                \hrule
                \begin{itemize}
                    \item aber wenn ein Akteur kompromittiert, dann SCHLECHT
                \end{itemize}
            }
        \end{column}
        
        \begin{column}{0.4\textwidth}
            \vspace{1em}
            \begin{figure}
                \includegraphics[height=4cm]{./assets/web_of_trust.drawio.pdf}
                \caption{Web of Trust}
            \end{figure}
        \end{column}
    \end{columns}
\end{frame}


\subsection{Erweiterung: Wertschöpfungsketten}

\begin{frame}{Erweiterung: Wertschöpfungsketten \footnotesize\cite{bothSolidBasedB2BData2025}}
    \begin{itemize}
        \item automatisierte Datenübertragung essenziell für Partner:in in Wertschöpfungsketten
        \item weitere Anforderungen
        \begin{itemize}
            \item garantierte, nachvollziehbare Einhaltung rechtlicher Rahmenbedingungen
            \item Constraints für Data Sharing
        \end{itemize}
    \end{itemize}

    \begin{figure}
        \includegraphics[width=\textwidth]{./assets/example_horizontal.drawio.pdf}
        \caption{Beispiel: Datenspende}
    \end{figure}
\end{frame}


\begin{frame}{Erweiterung: Wertschöpfungsketten II \footnotesize\cite{bothSolidBasedB2BData2025}}
    \addtocounter{figure}{-1}
    \begin{figure}
        \includegraphics[width=\textwidth]{./assets/example_horizontal.drawio.pdf}
        \caption{Beispiel: Datenspende}
    \end{figure}
    
    \vspace{-1em}

    \begin{itemize}
        \item Einführung zusätzlicher Metadaten
        \begin{itemize}
            \item \emph{Data Processing Purpose} $P$
            
            \note{
                \begin{itemize}
                    \item Weitergabe muss sich trotzdem an ursprüngl. Zweck halte
                \end{itemize}
            }

            \item Verstecken der Datenquelle, Angabe über Weitergabe
        \end{itemize}

        \item<2-> weitere Validierung notwendig
    \end{itemize}

    \note{
        \hrule
        \begin{itemize}
            \item weiterer Schritt in Richtung E2E-B2B-Wertschöpfungsketten
            \item Möglichkeit und Notwendigkeit zur Erweiterung von Solid
        \end{itemize}
    }

    % TODO: Stichpunkte zum Erzählen
\end{frame}
