% !TeX root = ../presentation.tex

\section{Potenziale und Ziele}

\begin{frame}{Vertrauen}
    \textbf{Warum kein Vertrauen?}
    \begin{itemize}
        \item Angst vor Datenmissbrauch oder unberechtigter Weitergabe~\cite{mollerIndustrialDataEcosystems2024}
    \end{itemize}

    \pause
    \textbf{Was würde helfen?}
    \begin{itemize}
        \item Verwaltung von Daten durch Nutzende unabhängig von Anwendungen
        \item Daten im Besitz der Nutzenden
        \item Kontrolle über Daten und Zugang

        \pause
        \item[$\Rightarrow$] Datensouveränität \& Datenschutz
    \end{itemize}
\end{frame}

\begin{frame}{Vertrauen II}
    \textbf{Wem kann ich vertrauen? Ist das wirklich dieser Akteur?}
    \begin{itemize}
        \item[$\to$] Authentifizierungsverfahren
        \item[$\to$] Vertrauen in geteilte Daten

        \pause
        \item[$\Rightarrow$] basierend auf Netzwerk $\to$ \emph{Web of Trust}
        \item[$\Rightarrow$] ad"=hoc Data Sharing möglich
    \end{itemize}
\end{frame}

\begin{frame}{Vertrauen III}
    \textbf{Einhaltung gesetzlicher Maßnahmen}
    \begin{itemize}
        \item technische Maßnahmen notwendig

        \pause
        \item physische und logische Trennung von Datenspeichern je nach Kontext
        \begin{itemize}
            \item bspw. Medizin"= getrennt von Bewerbungsdaten
        \end{itemize}

        \pause
        \item[$\Rightarrow$] Dezentralisierung
        \item[$\Rightarrow$] Interoperabilität auf Daten"= statt Anwendungsebene
    \end{itemize}
\end{frame}


\begin{frame}{Aktualität und Verfügbarkeit von Daten}
    \textbf{Warum veraltet und schlecht verfügbar?}
    \begin{itemize}
        \item Angst vor Datenmissbrauch und unberechtigter Weitergabe $\to$ mangelndes Vertrauen

        \pause
        \item[$\to$] Datensouveränität und Datenschutz
        \item[$\to$] höhere Wahrscheinlichkeit zur Speicherung von mehr und diverseren Daten

        \pause
        % Verfügbarkeit?
        \item[$\to$] Interoperabilität, offene Standards

        \pause
        \item[$\Rightarrow$] Aktualität und Verfügbarkeit von Daten
    \end{itemize}
\end{frame}


\begin{frame}{Kosten, Effizienz und Geschwindigkeit}
    \textbf{Was ist so teuer?}
    \begin{itemize}
        \item aufwendige Integrationen durch unterschiedliche Datenstrukturen
        \item wiederholte Schritte, bspw. Implementierung von Standardfunktionalitäten
    \end{itemize}

    \pause
    \textbf{Was würde helfen?}
    \begin{itemize}
        \item Auslagerung von Standardfunktionalitäten, Wiederverwendbarkeit, Interoperabilität
        \item[$\Rightarrow$] Automatisierung $\Rightarrow$ Kostensenkung, schnelle Prozesse
        \item[$\Rightarrow$] ad-hoc Zusammenschaltung von Geschäftsprozessen
    \end{itemize}
\end{frame}


\begin{frame}{Zugänglichkeit}
    \begin{itemize}
        % wenn oben erfüllt, dann können wir ... entgegenwirken
        \item[?] hohe Kosten und Aufwand
        \item[?] wenig verfügbare Daten, proprietäre Datenformate
    \end{itemize}

    \pause
    \begin{itemize}
        \item Automatisierung $\to$ niedrigere Kosten und Aufwand
        \item breitere Verfügbarkeit von Daten, Interoperabilität, offene Standards

        \pause
        \item[$\Rightarrow$] niedrigere Einstiegsbarrieren
        \item[$\Rightarrow$] Zugänglichkeit $\Rightarrow$ Wettbewerb, Innovation
    \end{itemize}
\end{frame}


\begin{frame}{Wie erreichen wir das?}
    \Large Kooperation verschiedener Akteure in einem vertrauenswürdigen,
           offenen, interoperablen System für Data Sharing

    % hier setzt Konzept Data Spaces an
    % ermöglicht multilaterales, sicheres Data Sharing
    % Garantie von Datensouveränität
\end{frame}
