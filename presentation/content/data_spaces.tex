% !TeX root = ../presentation.tex

\section{Data Spaces}

\begin{frame}{Inter-Organisational Information Systems (IOIS) \footnotesize\cite{mollerIndustrialDataEcosystems2024}}
    \begin{columns}
        \begin{column}{0.6\textwidth}
            \begin{itemize}
                \item \alert{bilaterale} Beziehungen, bspw. Lieferketten
                \item tiefe Integration, automatisiertes Data Sharing
                \item zweckgebunden, bspw. Koordinierung und Optimierung von Lieferketten
                % \item Mittel zum Zweck
                
                \item<2-> mangelndes Vertrauen, streng formalisierte Nutzungsrichtlinien
                \item<2-> keine Daten teilen? $\to$ Ineffizienz $\to$ Verlust
                \item<2-> Herausforderungen bei Skalierung % Datenaustausch, Kontrolle
            \end{itemize}
        \end{column}
        
        \begin{column}{0.4\textwidth}
            \begin{figure}
                \includegraphics[height=0.5\textheight]{./assets/iois_architecture.drawio.pdf}
                \caption{IOIS Architektur}
            \end{figure}
        \end{column}
    \end{columns}
\end{frame}


\begin{frame}{Data Intermediaries \footnotesize\cite{mollerIndustrialDataEcosystems2024}}
    \begin{columns}
        \begin{column}{0.6\textwidth}
            \begin{itemize}
                % \item dynamisch, geteilter Zweck
                \item Gleichgewicht zwischen erhaltenem und gegebenem Aufwand
                \item Daten als strategische Ressource % statt Mittel zum Zweck
                % \item ermöglichen neue Geschäfte und Optimierung von Prozessen
                
                \item<2-> Interaktion und Kooperation \alert{multilateraler} Akteure
                \item<2-> Data User, Data Provider, Data Intermediary
                
                \item<3-> offener, dynamischer Datenaustausch
                \item<3-> Netzwerkeffekte, Förderung von Innovation
            \end{itemize}
        \end{column}
        
        \begin{column}{0.4\textwidth}
            \begin{figure}
                \centering
                \includegraphics[height=0.5\textheight]{./assets/industrial_de_architecture.drawio.pdf}
                \caption{Industrial Data Ecosystems mit Data Intermediary}
            \end{figure}
        \end{column}
    \end{columns}
\end{frame}


\begin{frame}{Data Spaces \footnotesize\cite{mollerIndustrialDataEcosystems2024}}
    \begin{columns}
        \begin{column}{0.6\textwidth}
            \begin{itemize}
                \item Vereinen von IOIS und Data Intermediaries
                \item dezentrale Speicherung von Daten bei Provider
                
                % TODO: ??? --> Kapselung, DS ist auch nur Data Provider, Kapselung
                % merkt man nicht, wenn DS größer wird
                % mehr Worte, was es meint --> Adapter / Mediator Pattern (weitergeben oder noch Kontrolle?)
                % auslagern
                \item<2-> \emph{Data Space Connectors} für bilateralen Datenaustausch % vgl. IOIS
                \item<2-> Zusammenbringen von Data User und Provider % vgl. Data Intermediary
                
                \item<3-> technische Garantie von Datensouveränität
                % -> Überwindung betriebl. Barrieren
                % Kontrolle über Zugriff und Verwendung bei Provider
                \item<3-> geteilter Raum für vertrauenswürdiges Data Sharing $\to$ Optimierung, Innovation
            \end{itemize}
        \end{column}
        
        \begin{column}{0.4\textwidth}
            \begin{figure}
                \includegraphics[height=0.5\textheight]{./assets/data_space_architecture.drawio.pdf}
                % TODO: Label einbauen
                \caption{Data Space Architektur}
            \end{figure}
        \end{column}
    \end{columns}
\end{frame}


\begin{frame}[c]{Data Spaces II \footnotesize\cite{mollerIndustrialDataEcosystems2024}}
    % verteilt by Design --> Daten bleiben bei Quelle, d.h. Data Provider
    % Zugang nur gewährt, wenn notwendig

    % TODO: mehrere Personen
    % jeder bringt seinen Teil mit stat großer DB
    % Duplikate mit erklären
    \vspace{1.5em}
    \begin{figure}
        \includegraphics[height=0.6\textheight]{./assets/central_vs_decentral.drawio.pdf}
        \caption{Symbolbild: zentralisierte vs. dezentralisierte Datenspeicherung}
    \end{figure}
\end{frame}


\begin{frame}{Data Spaces III \footnotesize\cite{mollerIndustrialDataEcosystems2024}}
    \begin{columns}
        \begin{column}{0.6\textwidth}
            \begin{itemize}
                \item dezentralisierte Datenspeicherung
                \begin{itemize}
                    \item[$\to$] verteilt \emph{by Design} %, Datenredundanz % TODO: Datenredundanz?
                \end{itemize}
                
                \item<2-> Datenintegration auf semantische Ebene
                \begin{itemize}
                    \item[$\to$]<2-> kein einheitliches Daten"=Schema notwendig
                \end{itemize}
                
                % TODO: globaler DS sehr schwer, Datenstrukturen sehr spezif. für Teilbereich, aber trotzdem verknüpft
                % jede Domäne hat noch weitere Daten, interessiert andere Seite nicht
                % daher eigene DS, die genau diesen Teilbereich machen, starke Verknüpfung
                % über DS hinweg nur schwache Verknüpfung, bspw. Produkte oder Krankheitsstatus für Arzt <-> AG
                % daher DE
                % hier auch DS-Connectors

                \item<3-> Verschachtelung / Überlappung von Data Spaces
                \begin{itemize}
                    \item[$\to$]<3-> \emph{Data Ecosystems}
                    % um einen oder mehrere föderierte Data Spaces
                    % technische Integration über Schnittstellen
                    % Datensouveränität & Verhinderung großer Daten-Silos: verschachtelt && überlappend
                \end{itemize}
                
                \item<4-> Integration über Schnittstellen
                \item<4-> Erreichen gemeinsamer Ziele
            \end{itemize}
        \end{column}

        \begin{column}{0.4\textwidth}
            \only<3->{
                \begin{figure}
                    \includegraphics[height=0.5\textheight]{./assets/data_ecosystem_architecture.drawio.pdf}
                    \caption{Data Ecosystem}
                \end{figure}
            }
        \end{column}
    \end{columns}
\end{frame}


\begin{frame}{Data Spaces IV \footnotesize\cite{mollerIndustrialDataEcosystems2024}}
    % TODO: prinzipiell überhaupt eine Chance, dass ich an benötigte Daten komme
    % --> Daten-Lieferketten, muss zu best. Domäne kommen
    % brauche Kontrollen --> DS Connectors
    % Beispiel heraussuchen

    \begin{itemize}
        \item flexible betriebliche Strukturen
        \item Zugang nur für bestimmte Akteure $\to$ \emph{Trusted Pool}
        \item sicheres, vertrauenswürdiges Data Sharing
        
        \pause
        \item Einbettung in Data Ecosystem
        \item \emph{Data Space Member} vs. \emph{Data Ecosystem Party}
        % DSM: direkt technisch in DS eingebunden
        % DEP: nur indirekter Zugriff über Data Space Connector (Schnittstelle zu DSM)
        
        \pause
        \item Governance"=Maßnahmen auf allen Abstraktionsebenen
        % Erfüllung rechtlicher Rahmenbedingungen
        % Ebene von DE, DS oder Use Case
    \end{itemize}
    
    % Data Spaces erfüllen am meisten Kriterien
    % Wie kann das funktionieren? --> Social Linked Data
    
    % TODO: Überall Solid nennen, formally known as Social Linked Data
    % daher kommt das
\end{frame}
