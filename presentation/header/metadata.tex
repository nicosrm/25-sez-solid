% !TeX root = ../presentation.tex

\title{Data Spaces am Beispiel von Solid}

% \subtitle[SUBTITLE]
%     {FORMATTED\\SUBTITLE}

\date{07. Juli 2025}
\author[N. Schramm]{Nico Schramm}

\titlegraphic{\flushright\includegraphics[width=2cm]{./assets/htwk_logo.png}}

\newcommand{\module}{Innovative Methoden für Software Engineering der Zukunft}
\newcommand{\prof}{Prof. Dr. Andreas Both}

\newcommand{\faculty}{Fakultät Informatik und Medien}
\newcommand{\university}{Hochschule für Technik, Wirtschaft und Kultur Leipzig}

\institute{\module\\
  \prof\\~\\
  \faculty\\
  \university}

\newcommand{\repourl}{https://github.com/nicosrm/25-sez-solid}

\hypersetup{
  % PDF metadata
  pdfauthor={Nico Schramm},
  pdftitle={Data Spaces am Beispiel von Solid},
  pdfsubject={Präsentation},
  % hyperref colouring
  colorlinks,
  allcolors=.,
  urlcolor=blue,
  citecolor=cyan
}
